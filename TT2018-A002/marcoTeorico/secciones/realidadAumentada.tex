\section{Realidad Aumentada}

\subsection{Definición}
La realidad aumentada (AR) es una aproximación visual interactiva en tiempo real en la cual objetos virtuales son añadidos al entorno real, normalmente en la parte superior de un video, usando gráficos de computadora o móviles.\cite{B04} \par
Básicamente la realidad aumentada se refiere a que imágenes virtuales hechas por computadora sean mezcladas con la vista real para crear una viisión con elementos agregados. Su principal fundamento es mezclar la realidad con la realidad virtual. Aunque no sólo se trata de mezclar la realidad virtual, también pueden mezclarse elementos como audio, sensaciones, tacto, olores, y gusto los cuales son superpuestos sobre el mundo real para producir un \textbf{entorno de realidad aumentada}.\cite{B05}
Ésta tecnología no es nueva, tuvo sus orígenes en la década de los 90s\cite{B04}, y su primer aparición a nivel mundial fue en Octubre de 1998 en The FIrst International Workshop on AR (IWAR'98) en San Francisco\cite{B05}.\par

\subsection{Tipos de realidad aumentada}

\subsection{Impacto de la realidad aumentada}
Desde su aparición en los 90s, la realidad aumentada ha sido una tecnología de punta la cual sólo estaba disponible en algunos laboratorios y de forma remota al público en general. \par
Más allá de ser una útil técnica de visualización, actualmente es usada en muchos sectores como la ingenería, la medicina, la robótica, la milicia, la educación, entretenimiento, etc.\par
La realidad aumentada se ha vuelto una tecnología bastante común, y esto es debido principalmente a la popularidad que tienen los dispositivos inteligentes portátiles (como smartphones y tablets), que ya tienen integradas características como cámaras de video, resolución de alta definición y gran poder computacional que se adecúa al procesamiento de datos relacionados a la realidad aumentada.\par
Hoy, con el software apropiado, la mayoría de los usuarios de dispositivos inteligentes pueden usar ésta tecnología para múltiples propósitos sin tener que comprar otros dispositivos de costos elevados como los HDM (Head Mounte Displays). Adicionalmente con el anuncio de Google Glasses (AR Glasses), se puede inferir que el siguiente nivel de la realidad aumentada será la implementación de la realidad aumentada en pupilentes. Sin embargo puede asumirse que los dispositivos inteligentes serguirán siendo la herramienta dominante para el propósito general de la realidad aumentada en el futúro.\par
Como una poderosa herramienta de visualización, la realidad aumentada es aplicable en muchas áreas del diseño. A través de la realidad aumentada, las propuestas de diseño pueden ser examinadas con anticipación. En adición a esto, naturalmente la realidad aumentada mezcla elementos virtuales previamente renderizados con el entorno real, lo cual la convierte particularmente útil para las áreas del diseño que involucran construcción de entornos tales como el diseño urbano, la arquitectura y el diseño de interiores. Al reemplazar una porción del entorno físico con algún modelo tridimensional, la diferencia entre el estatus actual del entorno y el estatus similado se vuelve evidente, logrando que sea posible previsualizar un diseño, sin necesidad de llevarlo a cabo.\par
En el campo de la investigación del diseño de interiores, la realidad aumentada ha sido usada para simular objetos dentro de un espacio arquitectónico determinado. Estos objetos a menudo son muebles, accesorios o aparatos. Con el entorno físico capturado por un dispositivo en tiempo real y procesado de forma visual, el renderizado de objetos 3D es superpuesto.\cite{B15}

\subsection{Plataformas de realidad aumentada para dispositivos mófilves}
El impacto y la popularidad de la realidad aumentada han sido tan grandes, que actualmente existen varias plataformas para desarrollar tecnologías que usen realidad aumentada, ya sea para dispositivos móviles como smartphones y tablets, o para dispositivos integradores como los cascos de realidad virtual. Dentro de las plataformas más usadas podemos encontrar:
\begin{itemize}
	\item Wikitude
	\item ARKit
	\item Vuforia
	\item ARToolKit
	\item ARCore
\end{itemize}

\subsubsection{Wikitude}
Wikitude es una plataforma para el desarrollo de realidad aumentada (AR) para smartphones, tabletas y gafas inteligentes que nos permite una combinación de un mundo fisico y virtual. Wikitude nos proporciona potentes funciones en combinación con la realidad aumentada (AR) disponibles para los sistemas móviles Android, iOS, Google Glass, Epson Moverio, Vuzix M-100, Optinvent ORA1, PhoneGap, Titanium y Xamarin. Algunas de estas funciones son:\cite{B09}
\begin{itemize}
	\item \textbf{Escaneo de objetos para su reconocimiento}.- Wikitude proporciona un escaneo por medio de la cámara de un dispositívo móvil hacia los objetos reales, las últimas versiones de Wikitude proporciona una API  \textbf{Simultaneous Location  and Mapping SLAM}  que permite escaneos mas amplios como habitaciones. 
	
	\item \textbf{Seguimiento instantáneo}.- La tecnología Instant Tracking hace posible que las aplicaciones de realidad aumentada trabajen sin necesidad de un marcador y sobreponiendo los objetos en superficies reales basado en su propio seguimiento instantáneo gracias a su tecnologia \textbf{SLAM}. Ademas incluye una funcion llamada \textbf{SMART} que hace posible la compatibilidad con ARCore de Google o ARKit de Apple, esto dependerá de la compatibilidad del dispositívo.
	
	\item \textbf{Recomicimiento 2D}.- Esta tecnología interna de reconocimiento y seguimiento de imágenes de Wikitude funciona con imágenes para su reconicimiento, los desarrolladores pueden cambiar los angulos, dimensiones y la ubicación  dentro de la imagen del mundo real obtenida de la cámara del dispositívo.
	
	\item \textbf{Servicios de geolocalización}.- Esta tecnología nos permite combinar la realidad aumentada usando datos georreferenciados. Dependiendo del uso que se le de a la aplicación, la ubicación de su interés se apoya de la tecnologia GPS (sistema de posicionamiento global).Un ejemplo de este típo de aplicaiones es Pokemon GO. 
	
	\item \textbf{Reconocimiento multi-imagen}.- Esta característica permite el reconocimiento de varias imágenes simultáneamente. Una vez que se reconocen las imágenes, los desarrolladores pueden manipular modelos 3D, botones, videos, imágenes, entre otros y combinarlos con el mundo real, este tipo de reconocimiento múltiple de imágenes se puede usar para brindar interactividad a muchas aplicaciones.
	
	\item \textbf{Seguimiento extendido}.- Extended Tracking permite a los desarrolladores ir más allá de los objetivos de imágenes y objetos, los usuarios pueden continuar la experiencia de realidad aumentada moviendo libremente sus dispositivos sin la necesidad de mantener un marcador en la vista de la cámara. Esta función de igual forma se apoya de \textbf{SLAM]. 
	
	\item \textbf{Reconocimiento en la nube}.- El servicio de reconocimiento en la nube de Wikitude permite a los desarrolladores trabajar con miles de imágenes alojadas en la nube y trabaja con un tiempo de respuesta muy rápido.
	
	\item \textbf{Aumentado en 3D}.- Esta tecnología cargar y renderizar modelos 3D en la escena proporcionada por la camara del dispositívo, importandolos desde herramientas de renderizado como \textbf{Autodesk, Maya o Blender 3D} apoyadose de Unity3D para integrar el motor de aminacion por computadora.\cite{B16}
	
\end{itemize}

\noindent
Wikitude nos ofrece muchas vertientes para el desarrollo de realidad aumentada (AR), actualmente la convierten en una de las mejores tecnologias, cabe señalar que existe en su pagina oficial una version de prueba y se puede adquirir la licencia desde 2490 euros.\cite{B16} Algunos de los dispositivos compatibles con Wikitude se muestran en la siguiente tabla.

\begin{table}[]
	\begin{tabular}{|c|l|}
		\hline
		\textbf{Marca}              & \multicolumn{1}{c|}{\textbf{Modelo}}               \\ \hline
		\multirow{5}{*}{Google}     & Nexus 4+                                            \\ \cline{2-2} 
		& Nexus 5+                                           \\ \cline{2-2} 
		& Nexus 6P                                           \\ \cline{2-2} 
		& Nexus 10+                                          \\ \cline{2-2} 
    	& Google Glass                                       \\ \hline
		\multirow{1}{*}{Epson} & Epson Moverio BT-200        \\ \cline{2-2} \\ \hline
		\multirow{3}{*}{Apple}     & iPhone 4+               \\ \cline{2-2} 
		& Pad2X			                                    \\ \cline{2-2} 
		& iPod Touch 5th gen                                 \\ \hline 
		\multirow{1}{*}{Vuzix}         & Vuzix M100          \\ \cline{2-2} \\ \hline
		\multirow{5}{*}{Samsung} 	   & Galaxy S2+           \\ \cline{2-2} 
		& Galaxy S7, Galaxy S7 edge                          \\ \cline{2-2} 
		& Galaxy S8, Galaxy S8+                              \\ \cline{2-2} 
		& Galaxy S9, Galaxy S9+                              \\ \cline{2-2} 
		& Galaxy Tab S4                                      \\ \hline
	\end{tabular}
	
	\captionsetup{justification=centering}
	\caption*{Tabla 1. Listado de dispositivos compatibles con Wikitude}
\end{table}

\subsubsection{ARKit}
\subsubsection{Vuforia}
\subsubsection{ARToolKit}
\subsubsection{ARCore}
ARCore es la plataforma de Google para construir experiencias de realidad aumentada. A través de diferentes APIs, ARCore le da la habilidad a un dispositivo móvil para percibir su entorno, entender el mundo e interactuar con la información. Algunas de nuestras APIs están disponibles a través de Android y iOS para crear experiencias de realidad aumentada compartidas.\par
ARCore tiene nueve pilares básicos que permiten integrar contenido virtual en el entorno real, por medio de la cámara de un dispositivo móvil:

\begin{itemize}
	\item \textbf{Rastreo de movimiento}.- Cuando el celular se mueve a través del mundo real, ARCore usa un proceso llamado \textbf{Odometría concurrente y mapeo} o simplemente \textbf{COM}, para entender en qué posición se encuentra con respecto al entorno. ARCore detecta visualmente distintas características en las imágenes capturadas por la cámara para definir puntos llamados \textbf{puntos característicos}, a partir de los cuales virtualmente genera una maya de puntos, y los usa para computar el cambio físico de su posición. Esto se combina con mediciones realizadas por el teléfono para determinar la posición y la orientación de la cámara relativa al mundo en tiempo real.
	
	\item \textbf{Entendimiento ambiental}.- ARCore trata de identificar una maya de puntos coincidentes dentro de una misma superficies, como una mesa, el suelo o un muro, y hace que cada una de estas superficies detectadas esté disponible como un \textbf{plano}. ARCore también es capaz de detectar los límites de cada plano y convertir esto en información útil para las aplicaciones, de tal forma que sea posible posicionar objetos virtualmente sobre cada uno de estos planos. Debido a que ARCore usa puntos característicos para detectar planos, superficies sin texturas como parades blancas pueden no ser detectadas apropiadamente.
	
	\item \textbf{Estimación de luz}.- ARCore puede detectas información sobre la iluminación del entorno y proveer a la cámara de dispositivo móvil de una correción de color y gamma, para lograr una imágen óptima. Esta información permite variar la iluminación de una habitación y ver cómo la iluminación también varía en los elementos virtuales, aumentando la sensación de realismo.
	
	\item \textbf{Interacción con el usuario}.- Una vez que se ha definido la maya de puntos, ARCore permite que el usuario interactúe con el entorno visualizado en la cámara del dispositivo. El usuario puede hacer tap, o realizar gestos con los dedos sobre la pantalla del móvil, y ARCore tiene la capacidad de interpretar y transportar éstas acciones al entorno virtual, por ejemplo, el usuario puede agregar un objeto en alguna superficie al tocar la pantalla del dispositivo con el dedo, tras esto se hace una cálculo de las coordenadas relativas X, Y del plano donde se va a agregar el objeto, y finalmente el objeto es mostrado a través de la cámara.
	
	\item \textbf{Puntos de orientación}.- Los puntos de orientación permiten colocar objetos virtuales en superficies inclinadas. Cuando se está detectando la maya de puntos y se detectan superficies cercanas con diferente ángulo, ARCore se encarga de calcular el ángulo de inclinación de tal superficie.
	
	\item \textbf{Anclas y rastreables}.- La posición y orientación puede cambiar conforme ARCore mejora su entendimiendo del entorno. Cuando se quiere posicionar un objeto virtual, se requiere definir una ancla para asegurar que su posición sea rastreada en tiempo real. Los planos y puntos son un tipo de objeto especial denominado \textbf{rastreable}. Como su nombre sugiere, son objetos que ARCore estará rastreando a lo largo del tiempo. Es posible anclar objetos a rastrables, de tal forma que estos objetos virtuales van a conservar su posición relativa en el mundo, no importando si la cámara se mueve e inclusive si estos salen del foco de visualización.
	
	\item \textbf{Imágenes aumentadas}.- Las imágenes aumentadas permiten desarrollar aplicaciones de realidad aumentada que puedan responder a imagenes en 2D específicas como paquetes o posters de películas. Los usuarios pueden tener experiencias de realidad aumentada cuando ellos enfocan la cámara del celular a uno de estos elementos, por ejemplo, al enfocar la cámara el póster de una película se puede hacer que un personaje aparezca dentro del entorno virtual, y desaparezca cuando la cámara deje de enfocar a la imagen.
	
	\item \textbf{Compartir}.- \textbf{Cloud Anchors API}) permite crear experiencias de realidad aumentada colaborativas con otras personas. Con ésta API, un dispositivo puede enviar un ancla y puntos característicos a a la nube. Éstas anclas pueden ser compartidas con otros usuarios ya sea de Android o iOS. Esto habilita a las apps para renderizar los mismos objetos 3D asociados a tales anclas, permitiendo a los usuarios tener la misma experiencia de realidad aumentada de forma simultánea
	
\end{itemize}

\noindent
Dado que ARCore es una tecnología nueva (su primer publicación oficial fue en febrero de 2018 y su primer versión estable fue en agosto de 2018) la lista de dispositivos compatibles con ella no es tan amplia. En la \textit{Tabla 2} podemos observar una lista con los dispositivos que actualmente soportan ARCore.\cite{B14}

% Please add the following required packages to your document preamble:
% \usepackage{multirow}
\begin{table}[]
	\begin{tabular}{|c|l|}
		\hline
		\textbf{Marca}              & \multicolumn{1}{c|}{\textbf{Modelo}}               \\ \hline
		\multirow{2}{*}{ASUS}       & Zenfone AR                                         \\ \cline{2-2} 
		& Zenfone ARES                                       \\ \hline
		\multirow{4}{*}{Google}     & Nexus 5X                                           \\ \cline{2-2} 
		& Nexus 6P                                           \\ \cline{2-2} 
		& Pixel, Pixel XL                                    \\ \cline{2-2} 
		& PIxel 2, Pixel 2 XL                                \\ \hline
		\multirow{5}{*}{HDM Global} & Nokia 6                                            \\ \cline{2-2} 
		& Nokia 6.1 Plus                                     \\ \cline{2-2} 
		& Nokia 7 Plus                                       \\ \cline{2-2} 
		& Nokia 8                                            \\ \cline{2-2} 
		& Nokia 8 Sirocco                                    \\ \hline
		\multirow{4}{*}{Huawei}     & Honor 10                                           \\ \cline{2-2} 
		& nova 3, nova 3i                                    \\ \cline{2-2} 
		& P20, P20 Pro                                       \\ \cline{2-2} 
		& Porsche Design Mate RS                             \\ \hline
		\multirow{4}{*}{LG}         & G6                                                 \\ \cline{2-2} 
		& G7 ThiQ                                            \\ \cline{2-2} 
		& V30, V30+, V30+ JOJO                               \\ \cline{2-2} 
		& V35 ThinQ                                          \\ \hline
		\multirow{7}{*}{Motorola}   & Moto GS5 Plus                                      \\ \cline{2-2} 
		& Moto G6                                            \\ \cline{2-2} 
		& Moto G6 Plus                                       \\ \cline{2-2} 
		& Moto X4                                            \\ \cline{2-2} 
		& Moto Z2 Force                                      \\ \cline{2-2} 
		& Moto Z3                                            \\ \cline{2-2} 
		& Moto Z3 Play                                       \\ \hline
		\multirow{4}{*}{OnePlus}    & OnePlus 3T                                         \\ \cline{2-2} 
		& OnePlus 5                                          \\ \cline{2-2} 
		& OnePlus 5T                                         \\ \cline{2-2} 
		& OnePlus 6                                          \\ \hline
		\multirow{10}{*}{Samsung}   & Galaxy A5                                          \\ \cline{2-2} 
		& Galaxy A6                                          \\ \cline{2-2} 
		& Galaxy A7                                          \\ \cline{2-2} 
		& Galaxy A8, Galaxy A8+                              \\ \cline{2-2} 
		& Galaxy Note8                                       \\ \cline{2-2} 
		& Galaxy Note9                                       \\ \cline{2-2} 
		& Galaxy S7, Galaxy S7 edge                          \\ \cline{2-2} 
		& Galaxy S8, Galaxy S8+                              \\ \cline{2-2} 
		& Galaxy S9, Galaxy S9+                              \\ \cline{2-2} 
		& Galaxy Tab S4                                      \\ \hline
		\multirow{3}{*}{Sony}       & Xperia XZ Premium                                  \\ \cline{2-2} 
		& Xperia XZ1, Xperia XZ1 Compact                     \\ \cline{2-2} 
		& Xperia XZ2, Xperia XZ2 Compact, Xperia XZ2 Premium \\ \hline
	\end{tabular}

\captionsetup{justification=centering}
\caption*{Tabla 2. Listado de dispositivos compatibles con ARCore}
\end{table}


----------------------------------------------