\section{Realidad Aumentada}

\subsection{Definición}
La realidad aumentada (AR) es una aproximación visual interactiva en tiempo real en la cual objetos virtuales son añadidos al entorno real, normalmente en la parte superior de un video, usando gráficos de computadora o móviles.\cite{B04} \par
Básicamente la realidad aumentada se refiere a que imágenes virtuales hechas por computadora sean mezcladas con la vista real para crear una viisión con elementos agregados. Su principal fundamento es mezclar la realidad con la realidad virtual. Aunque no sólo se trata de mezclar la realidad virtual, también pueden mezclarse elementos como audio, sensaciones, tacto, olores, y gusto los cuales son superpuestos sobre el mundo real para producir un \textbf{entorno de realidad aumentada}.\cite{B05}
Ésta tecnología no es nueva, tuvo sus orígenes en la década de los 90s\cite{B04}, y su primer aparición a nivel mundial fue en Octubre de 1998 en The FIrst International Workshop on AR (IWAR'98) en San Francisco\cite{B05}.\par

\subsection{Tipos de realidad aumentada}

\subsection{Impacto de la realidad aumentada}
Desde su aparición en los 90s, la realidad aumentada ha sido una tecnología de punta la cual sólo estaba disponible en algunos laboratorios y de forma remota al público en general
Más allá de ser una útil técnica de visualización, actualmente es usada en muchos sectores como la ingenería, la medicina, la robótica, la milicia, la educación, entretenimiento, etc.

\subsection{Aplicaciones de la realidad aumentada}
La realidad aumentada es usada para sistemas de visión incorporada o HUD (Head Up Display) y herramientas de información como Sekai Camera, el cual es un software para la cámara del teléfono usado para desplegar información de los objetos a los que la cámara es enfocada, por ejemplo edificiones. Actualmente sólo está disponible en Japón.
Posicionar objetos virtuales dentro de un entorno real ayuda a entender las relaciones espaciales y las dimensiones, requeridas por ejemplo en el diseño de interiores, la cual es una aplicacion prominente de la realidad aumentada. 
\subsection{Plataformas de realidad aumentada para dispositivos mófilves}

\subsubsection{Wikitude}
\subsubsection{ARKit}
\subsubsection{Vuforia}
\subsubsection{ARToolKit}
\subsubsection{ARCore}


----------------------------------------------

%La mayoría de las veces asociamos los términos de realidad aumentada con la realidad virtual como si fueran lo mismo, sin embargo existen grandes motivos para detallar sus diferencias de software, hardware y de la interacción con los usuarios. \\.\par 

%La realidad aumentada (AR) nos permite sobreponer objetos en tercera dimensión sobre una imagen en tiempo real, obteniendo una mezcla de lo real con lo virtual, mejorando la percepción del mundo real de un usuario. Estas imagenes en tiempo real las podemos obtener con las camaras de smartphones, tablets, smart glasses y computadoras. [1] \\. \par

%Por otro lado la realidad virtual (VR) se define como un sistema informático que genera en tiempo real representaciones de una realidad es decir, un mundo virtual donde puede llegar a existir una interacción con el usuario. Muchas de estas simulaciones requieren de gafas de VR compatibles con smartphones o gafas especiales como el Gear VR desarrollado por Samsumg en colaboración con Oculus VR. [2][3] \\. \par

%A diferencia de la VR, la AR es una tecnología que complementa  la percepción  e interacción  con el mundo real y permite al usuario estar en un entorno aumentado con información generada por computadora. [1] \\. \par

%Actualmente, existen dos grandes representantes en estas tecnologías. Tal vez por criterios de marketing diríamos que Google Glass es el representante de la realidad aumentada, mientras que el Oculus Rift de Facebook sería el representante de la realidad virtual. [4]
