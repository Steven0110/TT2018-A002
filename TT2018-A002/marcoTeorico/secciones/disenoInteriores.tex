\section{Diseño de interiores}
\subsection{Definicion}
El diseño de interiores es una profesión en la cual soluciones creativas y técnicas son aplicadas dentro de una estructura para lograr la construcción de un entorno interno determinado. Éstas soluciones son funcionales, mejoran la calidad de vida de los ocupantes y son aestéticamente atractivas. Los diseños deben apegarse al código y normas requisitados, y fomentar los principios de sustentabilidad ambiental definidos por el edificio o empresa. El objetivo del diseño de interiores es lograr una armonía en los espacios que habitamos y dar confort al usuario de dichos espacios.\cite{B01} \par
El diseño de interiores sigue una metodología sistemática y coordinada que incluye investigación, análisis e integración de conocimientos dentro de un proceso creativo. Dentro ésta metodología podemos ubicar distintos servicios o etapas, dependiento de la complejidad del trabajo, en las cuales encontramos: definición de los requerimientos funcionales para los espacios de las habitaciones, planeación de espacios interiores, realización de planos de construcción, definición de especificaciones de ubicación, colores y acabados en piso, paredes, materiales, equipo, mobiliario y muebles, administración de contratos de fabricación o instalación, etc.\par
En Estados Unidos el diseño de interiores es la única rama del diseño que está sujeta a las regulaciones federales y la ley gubernamental.\cite{B02}
\subsection{El diseño de interiores a través de la historia}

\subsubsection{Mid Century Modern}
\subsubsection{Feng shui}
\subsubsection{Deconstructivismo}
\subsubsection{Diseño Orgánico}

\subsection{Fundamentos del diseño de interiores}

\subsection{Proceso del diseño de interiores}

\subsection{Teoría de colores en el diseño de interiores}
X\\.\par 
