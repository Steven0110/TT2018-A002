\section{Descripción general de la aplicación}
Este sistema móvil será capaz de  brindarle al usuario final una herramienta que le permita observar muebles virtuales y al mismo tiempo de percibir la realidad obtenida por una cámara,\par

La aplicación contará con un inicio de sesión inicial a través del cual se podrá acceder a los escenarios y proyectos de un usuario. También habrá una sección para registro de cuenta y otra para recuperación de contraseña.
Al iniciar sesión
Se observará un video en tiempo real de la información virtual sobrepuesta dentro del entorno real por medio de la cámara. Las idea es darle una perspectiva más exacta en forma, color y dimensión al mueble u objeto de nuestro agrado. También permitirá guardar estos entornos a través de una serie de fotografías (escenarios) los cuales a su vez conformarán proyectos. %Para concluir enlistaremos las principales caracterisitcas que tendra este proyecto.\par
%\begin{enumerate}[1.]
	%\item La aplicación debe ser un sistema móvil capaz de comunicarse con los servicios de AWS (Amazon Web Services). Esta plataforma nos ofrece todo lo necesario para la interacción y persistencia de la información. AWS será el encargado de alojar la información dentro de una instancia de RDS que tendrá un motor MySQL donde se encontraran los usuarios y la información de los muebles. Por otro lado AWS nos ofrece S3 para el alojamiento de las imagenes por medio de S3 Bucket. 
	%\item Los datos registrados por cada usuario dentro de la aplicación serán nombre, email y contraseña.\par
	%\item Cada imagen será alojada en formato base64. Por otro lado el tamaño en bytes de cada objeto dependerá de la complejidad, textura, colores, dimensiones y detalles que influyan significativamente en el proceso de renderización en un dispositivo anfitrión.\par
	%\item Los usuarios que interactúen con la aplicación deberá contar con los permisos necesarios para gestionar  la interacción con hardware o software disponible en el dispositívo.\par	
%\end{enumerate}
