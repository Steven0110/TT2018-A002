\section{Descripción general de la aplicación}
Este sistema móvil sera capaz de  brindarle al usuario final una herramienta que le permita observar muebles virtuales y al mismo tiempo de percibir la realidad obtenida por una cámara,\par

Se observará un video en tiempo real de la información virtual sobrepuesta dentro del entorno real por medio de la cámara. Las idea es darle una perspectiva más exacta en forma, color, dimensión nuestro mueble u objeto de agrado. Los resultados de obtener esta realidad mezclada es generar acierto, certidumbre y seguridad a las necesidades de un usuario con pretensiones de adquirirlo mediante una compra. Esta aplicación será enfocada al e-Commerce y los compradores potenciales serán mueblerías y establecimientos de venta de artículos de decoración y hogar. Para concluir enlistaremos las principales caracterisitcas que tendra este TT.\par
\begin{enumerate}[1.]
	\item La aplicación debe ser un sistema móvil capaz de comunicarse con los servicios de AWS (Amazon Web Services). Esta plataforma nos ofrece todo lo necesario para la interacción y persistencia de la información. AWS será el encargado de alojar la información dentro de una instancia de RDS que tendrá un motor MySQL donde se encontraran los usuarios y la información de los muebles. Por otro lado AWS nos ofrece S3 para el alojamiento de las imagenes por medio de S3 Bucket. 
	\item Los datos registrados por cada usuario dentro de la aplicación será un nombre, correo y contraseña.\par
	\item Cada imagen será alojada en formato base64. Por otro lado el tamaño en bytes de cada objeto dependerá de la complejidad, textura, colores, dimensiones y detalles que influyan significativamente en el proceso de renderización en un dispositivo anfitrión.\par
	\item Los usuarios que interactúen con la aplicación deberá contar con los permisos necesarios para gestionar  la interacción con hardware o software disponible en el dispositívo.\par	
\end{enumerate}
