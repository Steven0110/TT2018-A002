\section{Requerimientos de la aplicación}
La siguiente sección describimos los módulos, detallando la funcionalidad de cada uno y su flujo con sus posibles escenarios.\par

\subsection{Registro de cuenta}
\textbf{Actor :} \textit{Cliente} \par
\textbf{Descripción :} Los usuarios deberán registrase si desean interactuar por completo con todas las funciones disponibles de la aplicación. Este registro se basa en crear una cuenta con tu nombre, correo electrónico y una contraseña.
\subsection{Inicio de sesión}
\textbf{Actor :} \textit{Cliente} \par
\textbf{Descripción :} Los usuarios deberán acceder con sus datos de registro para tener acceso a toda la funcionalidad de la aplicación.
\subsection{Recuperar contraseña}
\textbf{Actor :} \textit{Cliente} \par
\textbf{Descripción :} Los usuarios que olviden su contraseña de acceso deberán solicitarla y posteriormente enviada a su correo electrónico

\subsection{Consultar muebles}
\textbf{Actor :} \textit{Cliente} \par
\textbf{Descripción :} Los usuarios podrán consultar los muebles disponibles dentro de un catalogo, donde sera el modelo visualizado en la realidad aumentada.

\subsection{Cambiar de color a mueble}
\textbf{Actor :} \textit{Cliente} \par
\textbf{Descripción :} Los usuarios podran cambiarle el color a un mueble.

\subsection{Crear escenario}
\textbf{Actor :} \textit{Cliente} \par
\textbf{Descripción :} Los usuarios podrán crear un escenario,consiste en una galería de fotos de un mueble dentro del interior de un espacio tomado por el smartphone.

\subsection{Tomar foto}
\textbf{Actor :} \textit{Cliente} \par
\textbf{Descripción :} Los usuarios podrán registrar capturas de su experiencia de la realidad aumentada de la aplicación dentro de un escenario creado.

\subsection{Visualizar escenario}
\textbf{Actor :} \textit{Cliente} \par
\textbf{Descripción :} Los usuarios podrán ver un escenario tomado, donde observaran las imagenes guardas dentro de su smartphone.