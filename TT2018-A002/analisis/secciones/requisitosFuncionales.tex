\section{Requerimientos de la aplicación}
La siguiente sección describimos los módulos, detallando la funcionalidad de cada uno y su flujo con sus posibles escenarios.\par

\subsection{Registro de cuenta}
\textbf{Actor :} \textit{Cliente} \par
\textbf{Descripción :} Los usuarios deberán registrase si desean interactuar por completo con todas las funciones disponibles de la aplicación. Este registro se basa en crear una cuenta con tu nombre, correo electrónico y una contraseña.
\subsection{Inicio de sesión}
\textbf{Actor :} \textit{Cliente} \par
\textbf{Descripción :} Los usuarios deberán acceder con sus datos de registro para tener acceso a toda la funcionalidad de la aplicación.
\subsection{Recuperar contraseña}
\textbf{Actor :} \textit{Cliente} \par
\textbf{Descripción :} Los usuarios que olviden su contraseña de acceso deberán solicitarla y posteriormente enviada a su correo electrónico

\subsection{Consultar muebles}
\textbf{Actor :} \textit{Cliente} \par
\textbf{Descripción :} Los usuarios podrán consultar los muebles disponibles dentro de un catalogo, donde sera el modelo visualizado en la realidad aumentada.

\subsection{Cambiar de color a mueble}
\textbf{Actor :} \textit{Cliente} \par
\textbf{Descripción :} Los usuarios podran cambiarle el color a un mueble.




\begin{enumerate}[1.]
\item El usuario debe validarse por medio de un login para poder realizar cualquier función en la aplicación
\item El usuario podra seleccionar un mueble desde un catálogo. 
\item El usuario podrá modificar el color del mueble a visualizar. 
\item Implementación de la aplicación móvil 
\item El usuario obtendrá una imagen del mueble sobre la realidad que observa.
\end{enumerate}