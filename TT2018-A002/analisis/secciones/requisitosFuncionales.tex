\section{Requerimientos de la aplicación}
En la siguiente sección los requerimientos que va a tener la aplicación, así como el actor relacionado con cada uno.\par

\subsection{Registro de cuenta}
\textbf{Actor:} \textit{Usuario} \par
\textbf{Descripción:} Los usuarios deberán registrarse si desean usar todas las funcionalides del sistema.

\subsection{Inicio de sesión}
\textbf{Actor:} \textit{Usuario} \par
\textbf{Descripción:} Los usuarios deberán acceder a una cuenta previamente registrada para tener acceso a toda la funcionalidad de la aplicación.

\subsection{Recuperar contraseña}
\textbf{Actor:} \textit{Usuario} \par
\textbf{Descripción:} Cuando un usuario olvide su contraseña podrá iniciar un proceso de recuperación de cuenta dentro de la misma aplicación, al final del cual, el usuario podrá generar una nueva contraseña y poder iniciar sesión de nuevo.

\subsection{Crear escenario}
\textbf{Actor:} \textit{Usuario} \par
\textbf{Descripción :} Los usuarios podrán crear un escenario, el cual consiste en un conjunto de fotos o videos donde se aprecia el entorno de realidad aumentada generado por el smartphone.

\subsection{Guardar información de escenarios}
\textbf{Actor:} \textit{Usuario} \par
\textbf{Descripción:} Los usuarios podrán guardar información de escenarios como el tipo de habitación que involucra, el tipo de muebles que tiene, el presupuesto del cliente, etc.

\subsection{Guardar información de proyectos}
\textbf{Actor:} \textit{Usuario} \par
\textbf{Descripción:} Los usuarios podrán guardar información de proyectos como los datos del cliente, observaciones de los proyectos, especificaciones del cliente, etc.

\subsection{Definición de presupuesto inicial}
\textbf{Actor:} \textit{Usuario} \par
\textbf{Descripción:} Los usuarios podrán definir un presupuesto al momento de crear un nuevo escenario, el cual irá disminuyendo conforme los muebles sean agregados al escenario. Durante la creación de un escenario se mostrará una alerta cuando se rebase el presupuesto inicial definido.

\subsection{Visualización de cotización}
\textbf{Actor:} \textit{Usuario} \par
\textbf{Descripción:} Al final de la creación de un escenario, los usuarios podrán consultar la cotización del mismo, esto significa, poder saber el precio total del escenario y de qué se compone ese precio. Este precio incluye IVA cumpliendo con la regla de negocio \textbf{BR5}.

\subsection{Crear proyecto}
\textbf{Actor:} \textit{Usuario} \par
\textbf{Descripción:} Los usuarios podrán crear proyectos, que no son más que un conjunto de escenarios relacionados.

\subsection{Visualizar escenario}
\textbf{Actor:} \textit{Usuario} \par
\textbf{Descripción:} Los usuarios podrán ver un escenario previamente creado, para poder visualizar su contenido.

\subsection{Tomar foto}
\textbf{Actor:} \textit{Usuario} \par
\textbf{Descripción:} Los usuarios podrán registrar capturas de su experiencia de la realidad aumentada de la aplicación, las cuales conformarán los escenarios.

\subsection{Grabar y almacenar video}
\textbf{Actor:} \textit{Usuario} \par
\textbf{Descripción:} Los usuarios podrán grabar videos de los entornos de realidad aumentada generados, los cuales junto con las fotos, conformarán los escenarios.


\subsection{Consultar muebles}
\textbf{Actor:} \textit{Usuario} \par
\textbf{Descripción:} Los usuarios podrán consultar los muebles disponibles dentro de un catalogo, donde se verá el modelo visualizado en la realidad aumentada.

\subsection{Clasificación de muebles}
\textbf{Actor:} \textit{Usuario} \par
\textbf{Descripción:} Los muebles estarán clasificados de acuerdo a las categorías y subcategorías que el usuario defina en la plataforma web.

\subsection{Colocación de muebles}
\textbf{Actor:} \textit{Usuario} \par
\textbf{Descripción:}  Los usuarios podrán colocar muebles en el entorno de realidad aumentada a partir de un catálogo.

\subsection{Cambio de posición de muebles}
\textbf{Actor:} \textit{Usuario} \par
\textbf{Descripción:}  Una vez colocado un mueble, los usuarios podrán moverlo de lugar seleccionándolo sobre la pantalla y moviéndolo a donde desee.

\subsection{Subir muebles renderizados a la nube}
\textbf{Actor:} \textit{Usuario} \par
\textbf{Descripción:}  Los usuarios podrán subir muebles renderizados a una plataforma en línea de tal forma que puedan ser usados en la aplicación.

\subsection{Resaltar mueble seleccionado}
\textbf{Actor:} \textit{Usuario} \par
\textbf{Descripción:} Los usuarios seleccionar un mueble tocando la pantalla del celular para realizar modificaciones del mismo. Este objeto se resaltará sobre los demás para que el usuario tenga la certeza de que está modificando el mueble que desea.

\subsection{Mostrar precios de muebles en escena}
\textbf{Actor:} \textit{Usuario} \par
\textbf{Descripción:} Cuando un usuario coloca un mueble, podrá observar su precio en pantalla.

\subsection{Eliminar muebles}
\textbf{Actor:} \textit{Usuario} \par
\textbf{Descripción:} Los usuarios podrán borrar del escenario muebles que previamente han sido colocados. 
