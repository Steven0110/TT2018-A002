
\section{Reglas de negocio}
En esta siguiente sección se definen las reglas de negocio que estarán implicadas en el desarrollo del sistema.\par
\subsection{BR1. Unicidad de usuarios}
El email de los usuarios registrados en el sistema deberá ser único e irrepetible, pues constituye parte del par de credenciales de autenticatión al sistema (email y contraseña). Por lo tanto no es posible registrar una nueva cuenta con un email que ya ha sido registrado previamente.

\subsection{BR2. Contraseña segura}
Las contraseñas de las cuentas que se registren deberán cumplir con lo siguiente:
\begin{itemize}
	\item Tener al menos un dígito.
	\item Tener al menos una letra mayúscula.
	\item Tener al menos una letra minúscula.
	\item Tener al menos un caracter especial.
	\item Tener al menos una longitud de 8 caracteres.
\end{itemize}

\subsection{BR3. Seguridad en recuperación de cuenta}
Debe haber un mecanismo que asegure que solamente el usuario de una cuenta pueda reestablecer su contraseña. El mecanismo es el siguiente:
\begin{itemize}
	\item El usuario solicita la recuperación de su cuenta.
	\item Se genera un código de recuperación y se asocia a su cuenta.
	\item El código de recuperación es enviado via email al correo registrado en la cuenta.
	\item Se le solicita al usuario que escriba el código recibido para confirmar su autenticidad.
	\item Se valida que el código ingresado por el usuario es el mismo que el generado previamente.
	\item El usuario podrá volver a definir su contraseña si la validación anterior es correcta.
\end{itemize}

\subsection{BR4. Organización de muebles}
Los muebles estarán organizados dentro de subcategorías, mismas que estarán contenidas en cateogrías, de tal forma que el menú de muebles de la aplicación esté en el siguiente orden jerárquico:
\begin{itemize}
	\item Categoría.
	\item Subcategoría.
	\item Mueble.
\end{itemize}
\subsection{BR5. Impuesto al Valor Agregado en cotizaciones}
Las obras realizadas por diseñadores de interiores profesionales normalmente requieren la emisión de una factura, lo cual implica elevar el precio de la obra un 16\% correspondiente al IVA.