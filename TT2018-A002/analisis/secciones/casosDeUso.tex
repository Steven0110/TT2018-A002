
\newpage
\section{Casos de uso}
El la siguiente sección se define y describen los Actores,funciones y esenarios, describiendo su funcionalidad e interacción dentro de la aplicación.\par
\vspace{5mm}
\begin{figure}[h!]
	\centering
	\includegraphics[width=15cm,height=17cm]{imagenes/analisis/casosDeUso.jpg}
	\caption{Casos de uso.}
	\label{fig:analogo}
\end{figure}  
\newpage

\subsection{CU1 Acceso}\par
Escenario de autenticación y acceso a la aplicación. Las entradas son un correo y una contraseña, ambos validados en la base de datos. Incluye una función que recupera la contraseña de los usuarios mediante el envío de un email.
\begin{figure}[h!]
	\centering
	\includegraphics[width=12cm,height=6cm]{imagenes/analisis/login.jpg}
	\caption{CU1 - Login.}
	
	\label{fig:analogo}
\end{figure}  
\subsection{CU2 Registro} \par
	El registro de la información de un usuario, los datos registrados son el nombre, correo electrónico y una contraseña. El pre-registro con la información sera consultada para que no se registren usuarios idénticos.
\begin{figure}[h!]
	\centering
	\includegraphics[width=12cm,height=6cm]{imagenes/analisis/registrarUsuario.jpg}
	\caption{CU2 - Registrar usuario.}
	\label{fig:analogo}
\end{figure} 
\newpage
\subsection{CU3 Muebles}  \par
Escenario donde un usuario consulta, cambia de color y selecciona un mueble para visualizarlo con realidad aumentada.
\begin{figure}[h!]
	\centering
	\includegraphics[width=12cm,height=6cm]{imagenes/analisis/seleccionarMueble.jpg}
	\caption{CU3 - Gestión de muebles.}
	\label{fig:analogo}
\end{figure}

\subsection{CU4 Crear escenario}\par
El actor podrá crear escenarios, cada escenario es una galería de captura de imágenes generadas con un mueble virtual y la realidad registrada por la cámara, incluye la función de eliminar imágenes antes de guardar el escenario.
\begin{figure}[h!]
	\centering
	\includegraphics[width=12cm,height=6cm]{imagenes/analisis/crearEscenario.jpg}
	\caption{CU4 - Crear escenario.}
	\label{fig:analogo}
\end{figure}

\newpage
\subsection{CU5 Visualizar escenario}\par
En esta parte el usuario podrá ver las imágenes de un escenario generado con anterioridad, podrá seleccionarlo desde una lista con sus escenarios disponibles
\begin{figure}[h!]
	\centering
	\includegraphics[width=12cm,height=6cm]{imagenes/analisis/visualizarEscenario.jpg}
	\caption{CU5 - Visualizar escenario.}
	\label{fig:analogo}
\end{figure}