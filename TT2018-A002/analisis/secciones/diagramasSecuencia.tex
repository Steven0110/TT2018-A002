\newpage
\section{Diagramas de secuencia}
A continuación se describe la interacción de nuestras entidades de negocio y el ciclo de vida que tendrán en la aplicación. También se detalla la interacción,mensajes y la lógica implementada por cada uno de los diferentes escenarios.\par
\subsection{Registro}
El registro de un usuario nuevo consiste en capturar el correo, nombre y una contraseña. Esta captura sera tratada como un JSON el cual viajara atravéz de un método POST. Gateway se encargara de redirigir esta petición al servicio de registro. Este servicio viajara hacia RDS donde se enviara un INSERT para registrar los datos. Si el usuario y la contraseña se encuentra repetido, se enviara un mensaje de error, como se muestra en la figura 3.7
\begin{figure}[h!]
	\centering
	\includegraphics[width=14cm,height=8cm]{imagenes/analisis/DSregistrarUsuario.jpg}
	\caption{Diagrama de secuencia - Registrar usuario.}
	\label{fig:analogo}
\end{figure} 
\subsection{Login}
La validación de credenciales comienza con capturar el correo y contraseña  de un usuario previamente registrado. Esta información sera tratada como un objeto JSON, el cual se mandara por una método POST. La API Gateway direccionara la petición a un Servicio AWS y  comparara la información en la instancia creada en RDS de un motor de base de datos MySQL. RDS se encargara de generar la conexión y consulta a la base de datos. La petición regresara con un mensaje si hubo coincidencia, de lo contrario se enviara un error a la interfaz de usuario.
\begin{figure}[h!]
	\centering
	\includegraphics[width=14cm,height=8cm]{imagenes/analisis/DSacceso.jpg}
	\caption{Diagrama de secuencia - Registrar Acceso.}
	\label{fig:analogo}
\end{figure} 