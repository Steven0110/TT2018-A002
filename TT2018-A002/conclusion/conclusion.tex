\chapter{Conclusiones}

La aplicación obtuvo bastantes modificaciones en comparación a lo planeado desde el Protocolo de Trabajo Terminal que nos ayudaron a atacar de mejor forma nuestro objetivo general.\par
Al observar el proceso de diseño de interiores sin usar ARF (ver figura x.xx) y al usar ARF (ver figura x.xx), notamos un menor número de etapas, asimismo hay etapas que se realizan de forma paralela mientras se usa la aplicación (toma de requerimientos, desarrollo de propuesta y definición del alcance).\par
La toma de requerimientos se vuelve una tarea más sencilla porque el cliente se encuentra presente durante la generación del escenario y puede ver los resultados en tiempo real a través de la realidad aumentada permitiéndole definir mejor sus necesidades. De igual forma el cliente al estar supervisando el desarrollo de la propuesta, logra que al término de la creación del escenario ya se cuente con su aprobación.\par
En el proceso normal del diseño de interiores, el análisis presupuestal se obtenía antes de elaborar la propuesta de forma manual entre el cliente y el diseñador, por el contrario, con ARF tras crear un escenario se obtiene automáticamente una cotización que incluye el costo unitario de cada mueble usado en escena y la suma de global de estos costos.\par
Por lo anterior, ARF es una aplicación que hace del diseño de interiores un proceso más fácil y rápido.