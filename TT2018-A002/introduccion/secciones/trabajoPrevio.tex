\section{Trabajo previo}
Las aplicaciones y proyectos que abordan el problema anteriormente descrito son:

\begin{enumerate}
	\item Canvas (iOS).
	\item Amazon App.
	\item Fingo.
	\item Ikea Place.
	\item TT 2012-B043. Realidad aumentada aplicada a la decoración de interiores.
\end{enumerate}

De forma colectiva, en tales aplicaciones pudimos notar las siguientes características:

\begin{enumerate}
	\item El usuario puede escanear una habitación en formato tridimensional incluyendo los muebles y objetos que haya en ella.
	\item El usuario puede exportar el escaneo tridimensional de una habitación para usarlo en AutoCAD.
	\item Mediante realidad aumentada el usuario puede posicionar un objeto a donde enfoque la cámara del celular.
	\item Existe una posición relativa de los objetos, es decir, si el celular se mueve el objeto permanece en la misma posición.
	\item Se requiere un hardware especial además del dispositivo móvil.
\end{enumerate}

Cabe destacar que Fingo y el TT 2012-B043 utilizan marcadores físicos, colocados en el suelo, sobre los cuales se superponen los objetos tridimensionales, lo cual limita su uso, dado que son dependientes de un elemento externo.\par
Por otro lado, encontramos características que consideramos importantes para resolver el problema planteado, pero ninguna de las aplicaciones anteriores las posee, como son:


\begin{enumerate}
	%\item No están enfocadas a e-Commerce
	\item No existe un gran repertorio de submodelos de objetos
	\item No poseen valores agregados en los objetos en general, por ejemplo, que se muestre el costo de un producto.
	\item No muestra presupuestos generales que indiquen los costos de los productos agregados al entorno de realidad aumentada, ni permite definir un presupuesto inicial que limite los objetos que se van a agregar.
	\item No permiten guardar información relacionada a los entornos de realidad aumentada generados.
\end{enumerate}

En la \textbf{\textit{Tabla 1}} podemos apreciar una comparación de las aplicaciones anteriores y la aplicación que planeamos hacer con base en las características previamente descritas:\par

% Please add the following required packages to your document preamble:
% \usepackage{graphicx}
% \usepackage[table,xcdraw]{xcolor}
% If you use beamer only pass "xcolor=table" option, i.e. \documentclass[xcolor=table]{beamer}
\begin{table}[]
	\resizebox{\textwidth}{!}{%
		\begin{tabular}{|l|l|l|l|l|l|l|}
			\hline
			\textbf{Características}       & \textbf{Canvas}                                 & \textbf{Tango}           & \textbf{Fingo}           & \textbf{Ikea Place}      & \textbf{TT 2012-B043}    & \textbf{Nuestra App}     \\ \hline
			Escaneo                        & \cellcolor[HTML]{BFBFBF}{\color[HTML]{C0C0C0} } &                          & \cellcolor[HTML]{BFBFBF} & \cellcolor[HTML]{FFFFFF} & \cellcolor[HTML]{BFBFBF} & \cellcolor[HTML]{BFBFBF} \\ \hline
			Exportar                       & \cellcolor[HTML]{BFBFBF}                        &                          &                          & \cellcolor[HTML]{FFFFFF} & \cellcolor[HTML]{FFFFFF} &                          \\ \hline
			Enfoque                        &                                                 & \cellcolor[HTML]{BFBFBF} &                          & \cellcolor[HTML]{BFBFBF} & \cellcolor[HTML]{BFBFBF} & \cellcolor[HTML]{BFBFBF} \\ \hline
			Posición relativa              &                                                 & \cellcolor[HTML]{BFBFBF} &                          & \cellcolor[HTML]{BFBFBF} & \cellcolor[HTML]{FFFFFF} & \cellcolor[HTML]{BFBFBF} \\ \hline
			Hardware externo               & \cellcolor[HTML]{BFBFBF}                        &                          & \cellcolor[HTML]{BFBFBF} & \cellcolor[HTML]{BFBFBF} & \cellcolor[HTML]{BFBFBF} &                          \\ \hline
			Variedad                       &                                                 &                          & \cellcolor[HTML]{BFBFBF} & \cellcolor[HTML]{FFFFFF} & \cellcolor[HTML]{BFBFBF} & \cellcolor[HTML]{BFBFBF} \\ \hline
			Diseños realistas de objetos   &                                                 &                          & \cellcolor[HTML]{BFBFBF} & \cellcolor[HTML]{BFBFBF} & \cellcolor[HTML]{FFFFFF} & \cellcolor[HTML]{BFBFBF} \\ \hline
			Valor agregado                 &                                                 &                          & \cellcolor[HTML]{BFBFBF} & \cellcolor[HTML]{BFBFBF} & \cellcolor[HTML]{FFFFFF} & \cellcolor[HTML]{BFBFBF} \\ \hline
			Presupuesto general            &                                                 &                          &                          &                          &                          & \cellcolor[HTML]{BFBFBF} \\ \hline
			Presupuesto inicial            &                                                 &                          &                          &                          &                          & \cellcolor[HTML]{BFBFBF} \\ \hline
			Guardar información del diseño &                                                 &                          &                          &                          &                          & \cellcolor[HTML]{BFBFBF} \\ \hline
		\end{tabular}%
	}
	\caption{Comparativo de aplicaciones sobre diseño de interiores}
	\label{comparativoestadodelarte}
\end{table}