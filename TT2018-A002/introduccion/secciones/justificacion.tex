\section{Justificación}
Cuando nos cambiamos de hogar, inevitablemente tenemos que afrontarnos con la tarea de decorar las habitaciones que hay en él. En éste punto, hacerlo no resulta tan complicado dado que partimos de una habitación vacía y ésta se convierte en un lienzo en blanco para nuestra imaginación. Al no haber objetos presentes, la percepción espacial de quien decora no se ve afectada, de tal forma que éste escenario facilita el diseño de interiores. Desafortunadamente no siempre tenemos la oportunidad de decorar una habitación cuando ésta se encuentra vacía, pues normalmente ya hay muebles y objetos decorativos en ella, entonces el proceso se resume a agregar nuevos objetos. Si nosotros escogemos un mueble que se ve agradable a simple vista, puede que, al momento de colocarlo en la habitación, no se encuentre en armonía con los demás objetos, lo cual es uno de los objetivos del diseño de interiores \cite{B01}. Incluso al no seguir los procesos fundamentales que el diseño de interiores requiere, como la planeación del espacio \cite{B02}, es posible que se tenga qué reiniciar todo proceso, lo cual es cansado, por el esfuerzo realizado al reorganizar los elementos de la habitación.\par 
Aunado a esto, puede llegar el punto donde quien decora la habitación, al final ya no desee el mueble, y realice un proceso de devolución de producto, si es que la tienda donde lo compró lo permite. Entonces la tienda pasa al domicilio donde se encuentre el producto para recogerlo o el usuario va a la tienda a entregarlo. De cualquier forma, se traduce en una pérdida económica y de tiempo.\par 
Todas éstas consecuencias se podrían evitar si realizamos un diseño de interiores que siga todas las etapas descritas en el diagrama de procesos anteriormente mostrado, pero por otro lado éste proceso es complejo, tardado y costoso.\par
Este proceso resulta sencillo para un diseñador de interiores titulado y/o certificado \cite{B02}, pero no para alguien que no tiene esa misma preparación y aquí es donde se pueden provocar pérdidas económicas y de tiempo por parte del cliente que compra un mueble y/o por parte de la tienda si se efectúa un proceso de devolución de producto dañando el prestigio de la tienda o sucursal asociada a la venta de estos muebles u objetos.\par
