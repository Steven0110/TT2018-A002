\subsection{Inicialización}
En ésta fase se describirá la linea base sobre la que se desarrollará el proyecto, esto significa, definir las herramientas y dispositivos con los que se realizará el desarrollo y testing, qué tecnologías vamos a usar y el planteamiento general del sistema.\par
\noindent
\\
\textbf{Herramientas y Tecnologías}:
	\begin{itemize}
			\item \textbf{Amazon Web Services.} Amazon Web Services (AWS) es una infraestructura de servicios en la nube usados para el desarrollo de sistemas. AWS tiene una gran cantidad de servicios útiles para el desarrollo de ARF de los cuales se usarán los siguientes para el desarrollo de ARF: S3 (almacenamiento), RDS (base de datos), Lambda (ejecución de código en la nube), API Gateway (comunicación con Lambda via internet).
			\item \textbf{Android 8 Oreo y Android 7 Nougat.} Estas son las versiones de Android que son compatibles con ARCore 1.5
			\item Android Studio 2.19
			\item \textbf{ARCore 1.5.} Es una librería para el desarrollo de aplicaciones con realidad aumentada de Google. Tras las pruebas de contexto se eligió usar ARCore en su versión 1.5 que es la versión estable más actual.
			\item \textbf{Blender 2.79.} Es un modelador y renderizador de objetos en 3D. Se usará para las primeras integraciones de ARCore donde utilizaremos modelos 3D de elaboración propia.
			\item \textbf{GitHub.} Es un sistema de control de versiones que sirve para tener el código fuente centralizado en repositorios y que varias personas puedan actualizar simultáneamente el código de forma rápida y sencilla. El código de ARF estará alojado en un repositorio de GitHub.
			\item \textbf{Git.} Es un cliente usado para la integración de GitHub en plataformas como Windows y GNU-Linux. Se usará para realizar la comunicación con GitHub.
	  		\item \textbf{JSON.} Es un formato para la estructuración de datos. Se usará debido a su facilidad de uso e integración con AWS.
	\end{itemize}
	\noindent
\textbf{Dispositivos}:
\begin{itemize}
	\item \textbf{Laptop.} Marca Acer, procesador Intel Core i3 6ta generación, 4GB de RAM, 1TB en disco duro, S.O. Debian 9.5
	\item \textbf{Laptop.} Marca HP, procesador Intel Core i5 6ta generación, 4GB de RAM, 1TB en disco duro, S.O. Windows 10
	\item \textbf{PC} Procesador AMD Ryzen 7 1800X, 16GB de RAM, 1TB en disco duro, S.O. Windows 10
	\item \textbf{Móvil.} Moto G6, procesador Snapdragon 450 a 1.8GHz, 3GB de RAM, 32GB de almacenamiento, S.O. Android 8.0
	\item \textbf{Móvil.} Moto G6+, procesador Snapdragon 630 a 2.2GHz, 4GB de RAM, 32GB de almacenamiento, S.O. Android 8.0	
\end{itemize}
\noindent
\textbf{Arquitectura}:
El backend de la aplicación se encontrará sobre la infraestructura de Amazon Web Services (AWS), debido a la alta escalabilidad que proporciona. La arquitectura contendrá un cluster RDS con MySQL, cinco Lambdas, un Bucket de S3 y una API Gateway (véase Figura 3.15).\par
Cada Lambda estará enfocada a una funcionalidad principal de la aplicación:\par
\begin{itemize}
	\item\textbf{Login}.- Encargada de toda la lógica y la seguridad informática relacionada con la autenticación en la aplicación.
	\item\textbf{Registrar cuenta}.- Permitirá registrar una nueva cuenta en el sistema, misma que permitirá guardar escenarios.
	\item\textbf{Recuperar cuenta}.- En caso de que se olvide la contraseña de la cuenta, ésta Lambda contendrá la lógica para recuperar la cuenta
	\item\textbf{Guardar escenario}.- Permitirá almacenar las imágenes que sean enviadas. Estas imágenes serán almacenadas en un bucket de S3 y asociadas a un escenario. Esta información será almacenada en la base de datos de MySQL que está montada sobre el cluster de RDS.
	\item\textbf{Ver escenario}.- Ësta Lambda permitirá recuperar la información e imágenes de un escenario especificado, con el fin de poder visualizarlo desde la ubicación.
\end{itemize}
\noindent
La aplicación podrá comunicarse a toda la infraestructura de AWS por medio de la API Gateway, que sirve como puerta de enlace tanto para entrar como para salir de AWS. Ésta comunicación se realizará a través del protocolo HTTPS, dada la facilidad de uso que proporciona, además de la capa de seguridad SSL que ya proporciona AWS durante la comunicación con la API Gateway. En la figura 4.25 puede observarse un diagrama de arquitectura del sistema.
\begin{figure}[H]
	\centering
	\includegraphics[width=15cm,height=15cm]{imagenes/desarrollo/arquitectura/ArchitecturaBackend.png}
	\caption{Arquitectura de Backend de ARF.}
	\label{fig:arqbackend}
\end{figure}
\par
Por otro lado se requerirá una base de datos que pueda almacenar los usuarios registrados y los escenarios que estos vayan creando. Para el tipo y cantidad de información que se requiere almacenar, una base de datos relacional cumple a ésta necesidad. En la figura  3.16 describe el diseño de la base de datos que se va a usar en ARF, misma que se encuentra en el clúster Amazon RDS mostrado en la figura 4.26.
\begin{figure}[H]
	\centering
	\includegraphics[width=10cm,height=7cm]{imagenes/desarrollo/arquitectura/ERD.png}
	\caption{Modelo Entidad-Relación usado para la base de datos de ARF.}
	\label{fig:arqbackend}
\end{figure}