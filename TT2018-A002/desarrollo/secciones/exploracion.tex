\section{Exploración}
\subsection{Establecimiento de interesados}
\subsection{Alcance}
\subsection{Establecimiento de proyectos}
\subsection{Pruebas de contexto}
Se realizaron pruebas de contexto definiendo las siguientes especificaciones:
\begin{itemize}
	\item Nombre del dispositivo en el que se realizó
	\item Fecha de prueba
	\item Versión de Sceneform SDK usada
	\item Versión de ARCore SDK
	\item Versión de Android
\end{itemize}
Por otro lado cada prueba se realizó tomando en cuenta las siguientes características:
\begin{itemize}
	\item \textbf{Posición Cardinal}.- Define si un objeto virtual puede visualizarse desde los cuatro puntos cardinales (norte, sur, este, oeste) y desde la parte superior del mismo, cuando la cámara gira alrededor del objeto mientras lo enfoca.
	\item \textbf{Tamaño relativo}.- Define si un objeto cambia su tamaño en el entorno virtual dependiendo de la distancia a la que se acerque o se aleje la cámara. Cuando la cámara se acerca, el objeto deberá aumentar su tamaño, y viceversa, como si se tratase de un objeto real.
	\item \textbf{Luminosidad}.- Define el grado de oscuridad de un objeto a determinada luz, es decir, cuando la luz en el ambiente real es alta, el objeto se verá iluminado, en caso contrario cuando haya escasa luz, el objeto se oscurecerá.
	\item \textbf{Superficie}.- Describe en qué superficies el objeto virtual es puesto, si fue posible su superposición en este material y qué comportamiento tiene en cada una de estas.
	\item \textbf{Memoria de objetos}.- Define si los objetos virtuales se conservan en la memoria cuando la cámara pierde su enfoque en ellos y los vuelve a enfocar. También describe si los objetos conservaron su posición tras el re-enfoque.
	\item \textbf{Capacidad máxima de objetos}.- Define el número de objetos virtuales que pueden ser mostrados en escena sin que el rendimiento de la aplicación caiga considerablemente.
	\item \textbf{Distancia}.- Define la distancia a la que se encuentra la cámara de un objeto virtual sin que éste desaparezca o sin que su resolución baje considerablemente.
\end{itemize}
\noindent

\subsubsection{MOTO G6 XT1925}
\begin{table}[!h]
	\centering
	\begin{tabular}{|c|c|}
		\hline
		\multicolumn{2}{|c|}{Especificaciones de prueba}   \\ \hline
		\textbf{DISPOSITIVO}              & Moto G6 XT1925 \\ \hline
		\textbf{FECHA}                    & 2018/08/25     \\ \hline
		\textbf{VERSIÓN DE SCENEFORM SDK} & V1.4.0         \\ \hline
		\textbf{VERSIÓN DE ARCORE SDK}    & V1.4.0         \\ \hline
		\textbf{VERSIÓN DE ANDROID}       & V8.0.0 (Oreo)  \\ \hline
	\end{tabular}
	\captionsetup{justification=centering}
	\caption{Especificaciones de prueba en Moto G6}
\end{table}

\textbf{Posición cardinal} \par
El objeto virtual se pudo apreciar con claridad desde los cuatro puntos cardinales y la vista superior. El ángulo de visualización del objeto al mover la cámara cambiaba a la perfección, dando una buena percepción de realismo.

%%IMAGENES DE PUNTOS CARDINALES


\textbf{Tamaño relativo} \par
Al acercar o alejar la cámara el objeto virtual variaba su tamaño de forma adecuada, como si el objeto realmente estuviera en la posición donde fue superpuesto.

%%IMAGENES DE TAAAÑO RELATIVO

\textbf{Luminosidad} \par
Al poner el objeto virtual en entornos con diferente cantidad de luz, la cantidad de luz en el objeto virtual también variaba. En un entorno con ausencia casi total de luz el objeto apenas era perceptible, mientras que en un entorno con bastante luz, el objeto se veía altamente iluminado.

%%IMAGENES DE LUMINOSIDAD

\textbf{Superficie} \par
Se probó posicionar el objeto virtual en cuatro superficies: concreto gris, concreto blanco, azulejo y vidrio.\par
Concreto gris.- El objeto se pudo posicionar a la perfección.\par
Concreto blanca.- El objeto no se pudo posicionar. La maya de puntos ni si quiera era detectada en ésta superficie debido a la ausencia de texturas.\par
Azulejo.- El objeto se pudo posicionar a la perfección.\par
Vidrio.- El objeto no pudo ser posicionado en ésta superficie debido a las propiedades reflejantes que posee.\par

%%IMAGENES DE SUPERFICIES

\textbf{Memoria de objetos} \par
Tras perder el enfoque de la cámara, al volverlo a tener, todos los objetos virtuales se volvieron a mostrar en el entorno virtual en la misma posición en la que habían sido puestos.

\textbf{Capacidad máxima de objetos} \par
Se colocaron 100 objetos virtuales en escena sin que la aplicación perdiera rendimiento. Todo funcionaba con total fluidez.

%%IMAGENES DE 100 OBJETOS

\textbf{Distancia} \par
Se colocó un objeto, después la cámara fue alejada hasta una distancia de \textbf{11.22m.} A esa distancia los objetos comenzaron a verse pixeleados, además comenzaron a desaparecer y reaparecer de forma intermitente.

%%IMAGENES DE DISTANCIA MÁXIMA



