\chapter{Desarrollo}

En éste capítulo se describirá el proceso de desarrollo del trabajo terminal de acuerdo a la metodología Mobile-D, de tal forma que estará divido en capítulos según las iteraciones que se vayan realizando.\par
Mobile-D es una metodología ágil iterativa basada en Extreme Programming. Consta de cinco etapas básicas: 
\begin{itemize}
	\item Exploración
	\item Inicialización
	\item Producción
	\item Estabilización
	\item Pruebas
\end{itemize}
\noindent
Solamente durante la primera iteración se realizarán las fases "Exploración" e "Inicialización", que sirven para definir la dirección del proyecto. Posteriormente "Producción" se realiza de forma iterativa, llévandose a cabo una y otra vez hasta terminar de construir o desarrollar el sistema. Las iteraciones y el entregable de cada una serán definidos durante la iteración inicial.
Finalmente en la fase "Estabilización" se integran los componentes que se hayan desarrollado y se documenta el proceso realizado en las iteraciones anteriores, para culminar con una última fase de "Pruebas" donde se asegura la calidad y funcionamiento del sistema para poder ser liberado, y en caso de que se requiera, se hagan correcciones.