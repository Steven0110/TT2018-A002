%%%%%%   TIPO DE DOCUMEN%TO: Reporte   %%%%%%
\documentclass[letterpaper,11pt]{report}
%%%%%%%%%%%%%%%%%%%%%%%%%%%%%%%%%%%%%%%%%%%%

%%%%%%%   PAQUETES   %%%%%%%
%\usepackage[spanish]{babel}
\usepackage{graphicx}
\usepackage{indentfirst}
\usepackage[table,xcdraw]{xcolor}
\usepackage[utf8]{inputenc}
\usepackage [margin=1in,includefoot]{geometry}
%\usepackage{colortbl}
\usepackage{array,booktabs,xcolor}  %arydshln,multirow
\usepackage{caption}
\usepackage{multirow}
\usepackage{enumerate}
\usepackage{float} 
\newcommand\VRule[1][\arrayrulewidth]{\vrule width #1}
%%%%%%%%%%%%%%%%%%%%%%%%%%%%

%%%%%%%%%%%%%%%%%%%%%%%%%%%%%%%%%%%%%%%%
%%%%%%%   INICIO DEL DOCUMENTO   %%%%%%%
%%%%%%%%%%%%%%%%%%%%%%%%%%%%%%%%%%%%%%%%
\begin{document}

    %%%%%%% Renombrar en espaNol %%%%%%
    \renewcommand\bibname{Bibliografía}
    \renewcommand{\figurename}{Figura}
    \renewcommand{\tablename}{Tabla} %Escribe Tabla en lugar de Cuadro
    \renewcommand{\listtablename}{\'Indice de tablas} %Escribe Indeice de tablas en lugar de Indice de cuadros
    \renewcommand{\chaptername}{Capítulo}
    \renewcommand*{\contentsname}{Contenido}
    \renewcommand\listfigurename{Lista de figuras}

    %%%%%%%   PORTADA   y RESUMEN   %%%%%%% Este es el contenido agregado de la secci\'on 2.

    %%%%%%%%%%%%%%%%%%%%%%%%%%%%%
%%%%%      PORTADA      %%%%%
%%%%%%%%%%%%%%%%%%%%%%%%%%%%%

\begin{titlepage}

    \centering %Todo centrado

    %%%%  LOGO DE LA ESCUELA   %%%%
    \includegraphics[scale=0.17]{imagenes/escom-ipn} %Imagen para portada
    %%%%  NOMBRE DE LA ESCUELA   %%%%
    \LARGE{\\ Instituto Polit\'ecnico Nacional}
    \LARGE{\\ Escuela Superior de C\'omputo}
    
    \vspace{1cm} %Espacio vertical

    %%%%  TITULO Y NÚMERO DE TRABAJO   %%%%
    \LARGE \textbf{Augmented Reality Furniture (ARF)}
    \LARGE {\\ TT2018-A002}

    \vspace{1cm} %Espacio vertical

    \LARGE \textit{Que para cumplir con la opción de titulación curricular en la carrera de:}
    \LARGE \textbf{\\ Ingeniería en Sistemas Computacionales}

    \vspace{1cm} %Espacio vertical

    %%%%   ALUMNOS   %%%%
   \textit{Presentan}\\
    Cabello Acosta Gerardo Aramis\\
    Carrillo Mendoza Martín Alejandro \\
    Del Pilar Morales Saúl

    \vspace{1cm} %Espacio vertical

    %%%%   Directores   %%%%
   \textit{Directores}\\
    M. en C. Vélez Saldaña Ulises. \bigskip Director 1 \\
    M. en C. José David Ortega Pacheco. \bigskip  Director 2
\end{titlepage} %incluye el archivo portada.tex
    %%%%%%%%%%%%%%%%%%%%%%%
%%%%    RESUMEN    %%%%
%%%%%%%%%%%%%%%%%%%%%%%

\begin{abstract}

  En este reporte se presenta la documentación técnica y marco teórico del Trabajo Terminal 2018-A002 titulado: \textbf{Augmented Reality Furniture (ARF)}, cuyo objetivo es desarrollar una aplicación móvil que permita crear entornos virtuales utilizando realidad aumentada en dispositivos móviles para facilitar el diseño de interiores.\par
  \textbf{Palabras clave:} Aplicación móvil, realidad aumentada, diseño de interiores.

\end{abstract}
 %Incluir resumen del documento (resumen.tex)

    %%%%%%%   INCLUIR ENCABEZADOS EN INDICES Y CAPITULOS   %%%%%%%
    \pagestyle{headings}

    %%%%%%%   NUMERACION EN CONTENIDO E INDICE DE TABLAS Y FIGURAS   %%%%%%%
    \pagenumbering{roman} %NUmeros romanos
    %\setcounter{page}{1} %Comienza en I por default, aquI se puedo modificar

    %%%%%%   INCLUIR CONTENIDO, INDICE DE FIGURAS E INDICE DE TABLAS   %%%%%%
    \tableofcontents
    \listoffigures
    \listoftables

    %%%%%%%   NUMERACION EN CAPITULOS   %%%%%%%
    \clearpage %Para iniciar con los arAbigos
    \pagenumbering{arabic} %Numeros arabigos
    %\setcounter{page}{1} %Comienza en 1 por default, aquI se puede modificar

    %%%%%%%   INCLUYE CAPITULOS Y SECCIONES   %%%%%%%
    \chapter{Introducci\'on}

Colocar una descripción de lo que contiene el capítulo
	   \section{Contexto de trabajo}

Actualmente vivimos en un entorno dominado por la tecnología; día a día se desarrollan nuevas herramientas con el fin de ayudar al ser humano a realizar tareas de una forma más fácil y eficiente, como los HMD (Head-mounted Display)\cite{B15}.


Por otro lado también nos encontramos en una época donde el diseño es un área de gran importancia en cualquier sector del mercado, por ejemplo, un buen diseño web en un sitio es fundamental lograr que un producto se logre vender o difundir, un buen diseño gráfico en campañas de marketing asegura más clientes; de igual forma nos encontramos con el diseño de interiores. Para esta última área se suelen contratar diseñadores de interiores profesionales para lograr que los espacios interiores de un inmueble consigan tal armonía que mejoren la calidad de vida de quienes lo habitan y además generen un impacto en las personas que usan éstas habitaciones.\par
El diseño de interiores profesional no sigue un proceso estandarizado, sin embargo a nivel general podemos ubicar las siguientes etapas:

\begin{itemize}
	\item El cliente acude con un diseñador de interiores profesional, quien acude al domicilio para conocer las condiciones del inmueble y conocer las necesidades y presupuesto del cliente.
	\item Conociendo estos dos puntos el diseñador hace un análisis de viabilidad que se compone de un análisis presupuestal y la definición del alcance del diseño.
	\item El diseñador le muestra esta información al cliente para que la apruebe, en caso de no ser así, el proyecto es considerado como no viable.
	\item De ser viable el diseñador de interiores genera un calendario previo de actividades donde programa el cumplimiento de los requerimientos del cliente en determinadas fechas.
	\item Si el cliente aprueba la propuesta de calendario, el diseñador se dispone a realizar el \textbf{proceso de propuesta de diseño de interiores} con el objetivo de mostrarle al cliente el resultado final previo de la obra antes de que sea realizada, esta propuesta puede mostrarse de varias formas como fotografías con fotomontajes, una explicación verbal con fotografías de apoyo (de muebles o habitaciones ya decoradas) o un modelado tridimensional del resultado de la obra; se suele usar un modelado tridimensional pues es lo más cercano a la realidad que se le puede mostrar al cliente. Si el cliente no aprueba la propuesta de calendario, se debe volver a realizar el análisis de viabilidad para ajustarse a las necesidades y presupuesto del cliente.
	\item El diseñador le muestra la propuesta de diseño al cliente y si esta es aceptada, se inicia el desarrollo de la obra con base en las fechas establecidas.
	\item Finalmente el diseñador entrega la obra al cliente y termina el proceso de diseño de interiores.
\end{itemize}

El proceso completo anteriormente descrito se puede observar en la figura 1.1 a través de un diagrama de proceso de negocio, en las figuras 1.2 y 1.3 se muestra el diagrama segmentado para una mejor lectura del mismo, y en la figura 1.4 puede observar el \textbf{proceso de propuesta de diseño de interiores}.\par
Teniendo a la mano una gran diversidad de herramientas tecnológicas, podemos usar estos elementos para lograr que el diseño de interiores sea más sencillo y rápido, tanto para un diseñador de interiores que vaya a realizar una obra para algún cliente, como para alguien que desee diseñar o rediseñar su propio inmueble.
\newpage

\begin{figure}[!htbp]
	\centering
	\includegraphics[width=20cm,angle=270,origin=c]{imagenes/marcoteorico/bpmn/proceso_full.jpg}
	\caption{Modelo del proceso de diseño de interiores (Completo).}
	\label{fig:bpmn_antes}
\end{figure}
\newpage

\begin{figure}[!htbp]
	\centering
	\includegraphics[width=19cm,angle=270,origin=c]{imagenes/marcoteorico/bpmn/proceso_01_01_left.png}
	\caption{Modelo del proceso de diseño de interiores (Segmento I).}
	\label{fig:bpmn_antes}
\end{figure}
\newpage

\begin{figure}[!htbp]
	\centering
	\includegraphics[width=19cm,angle=270,origin=c]{imagenes/marcoteorico/bpmn/proceso_02_01_left.png}
	\caption{Modelo del proceso de diseño de interiores (Segmento II).}
	\label{fig:bpmn_antes}
\end{figure}
\newpage

\begin{figure}[!htbp]
	\centering
	\includegraphics[width=16cm]{imagenes/marcoteorico/bpmn/subproceso.jpg}
	\caption{Subproceso de propuesta de diseño de interiores.}
	\label{fig:subproceso}
\end{figure}
\newpage
	   \newpage
\section{Problemática}
El diseño de interiores es un proceso que implica muchas etapas, algunas de ellas son de tal complejidad que necesitan de un diseñador de interiores profesional para completarse con éxito. Hay otras que son de gran lentitud como el desarrollo de las propuestas del diseño pues requieren incluso la elaboración de modelos tridimensionales. Lo anterior conlleva a un proceso de diseño de interiores lento y complejo, el cual si no es realizado de forma adecuada, puede provocar que se repitan algunas etapas hasta conseguir el resultado deseado, lo cual genera pérdidas económicas y de tiempo tanto para el usuario que desea un diseño de interiores como para el profesional encargado de realizar éste diseño.

   
	   \section{Trabajo previo}
Las aplicaciones y proyectos que abordan el problema anteriormente descrito son:

\begin{enumerate}
	\item Canvas (iOS).
	\item Amazon App
	\item Fingo
	\item Ikea Place
	\item TT 2012-B043. Realidad aumentada aplicada a la decoración de interiores
\end{enumerate}

De forma colectiva, en tales aplicaciones pudimos notar las siguientes características:

\begin{enumerate}
	\item El usuario puede escanear una habitación en formato tridimensional incluyendo los muebles y objetos que haya en ella
	\item El usuario puede exportar el escaneo tridimensional de una habitación para usarlo en AutoCAD
	\item Mediante realidad aumentada el usuario puede posicionar un objeto a donde enfoque la cámara del celular
	\item Existe una posición relativa de los objetos, es decir, si el celular se mueve el objeto permanece en la misma posición
	\item Se requiere un hardware especial además del dispositivo móvil
\end{enumerate}

Cabe destacar que Fingo y el TT 2012-B043 utilizan marcadores físicos, colocados en el suelo, sobre los cuales se superponen los objetos tridimensionales, lo cual limita su uso, dado que son dependientes de un elemento externo.\par
Por otro lado, encontramos características que consideramos importantes para resolver el problema planteado, pero ninguna de las aplicaciones anteriores las posee, como son:


\begin{enumerate}
	%\item No están enfocadas a e-Commerce
	\item No existe un gran repertorio de submodelos de objetos
	\item No poseen valores agregados en los objetos en general, por ejemplo, que se muestren las propiedades del producto, o que se puedan cambiar colores de los mismos.
	\item No muestra presupuestos generales que indiquen los costos de los productos agregados al entorno de realidad aumentada, ni permite definir un presupuesto inicial que limite los objetos que se van a agregar
	\item No permiten guardar información relacionada a los entornos de realidad aumentada generados
\end{enumerate}

En la \textbf{\textit{Tabla 1}} podemos apreciar una comparación de las aplicaciones anteriores y la aplicación que planeamos hacer con base en las características previamente descritas:\par

% Please add the following required packages to your document preamble:
% \usepackage{graphicx}
% \usepackage[table,xcdraw]{xcolor}
% If you use beamer only pass "xcolor=table" option, i.e. \documentclass[xcolor=table]{beamer}
\begin{table}[]
	\resizebox{\textwidth}{!}{%
		\begin{tabular}{|l|l|l|l|l|l|l|}
			\hline
			\textbf{Características}       & \textbf{Canvas}                                 & \textbf{Tango}           & \textbf{Fingo}           & \textbf{Ikea Place}      & \textbf{TT 2012-B043}    & \textbf{Nuestra App}     \\ \hline
			Escaneo                        & \cellcolor[HTML]{BFBFBF}{\color[HTML]{C0C0C0} } &                          & \cellcolor[HTML]{BFBFBF} & \cellcolor[HTML]{FFFFFF} & \cellcolor[HTML]{BFBFBF} & \cellcolor[HTML]{BFBFBF} \\ \hline
			Exportar                       & \cellcolor[HTML]{BFBFBF}                        &                          &                          & \cellcolor[HTML]{FFFFFF} & \cellcolor[HTML]{FFFFFF} &                          \\ \hline
			Enfoque                        &                                                 & \cellcolor[HTML]{BFBFBF} &                          & \cellcolor[HTML]{BFBFBF} & \cellcolor[HTML]{BFBFBF} & \cellcolor[HTML]{BFBFBF} \\ \hline
			Posición relativa              &                                                 & \cellcolor[HTML]{BFBFBF} &                          & \cellcolor[HTML]{BFBFBF} & \cellcolor[HTML]{FFFFFF} & \cellcolor[HTML]{BFBFBF} \\ \hline
			Hardware externo               & \cellcolor[HTML]{BFBFBF}                        &                          & \cellcolor[HTML]{BFBFBF} & \cellcolor[HTML]{BFBFBF} & \cellcolor[HTML]{BFBFBF} &                          \\ \hline
			Variedad                       &                                                 &                          & \cellcolor[HTML]{BFBFBF} & \cellcolor[HTML]{FFFFFF} & \cellcolor[HTML]{BFBFBF} & \cellcolor[HTML]{BFBFBF} \\ \hline
			Diseños realistas de objetos   &                                                 &                          & \cellcolor[HTML]{BFBFBF} & \cellcolor[HTML]{BFBFBF} & \cellcolor[HTML]{FFFFFF} & \cellcolor[HTML]{BFBFBF} \\ \hline
			Valor agregado                 &                                                 &                          & \cellcolor[HTML]{BFBFBF} & \cellcolor[HTML]{BFBFBF} & \cellcolor[HTML]{FFFFFF} & \cellcolor[HTML]{BFBFBF} \\ \hline
			Presupuesto general            &                                                 &                          &                          &                          &                          & \cellcolor[HTML]{BFBFBF} \\ \hline
			Presupuesto inicial            &                                                 &                          &                          &                          &                          & \cellcolor[HTML]{BFBFBF} \\ \hline
			Guardar información del diseño &                                                 &                          &                          &                          &                          & \cellcolor[HTML]{BFBFBF} \\ \hline
		\end{tabular}%
	}
	\caption{Comparativo de aplicaciones sobre diseño de interiores}
	\label{my-label}
\end{table}
	   \section{Solución propuesta}
Para mejorar el proceso de diseño de interiores, proponemos desarrollar una aplicación móvil que permita a los usuarios visualizar a través de la realidad aumentada muebles y objetos decorativos en una habitación, eliminando la necesidad de tenerlos físicamente en ella, la cual servirá como herramienta de apoyo en este proceso.

 
	   \section{Objetivo}
\subsection{Objetivo general}
Desarrollar una aplicación móvil que permita crear entornos virtuales utilizando realidad aumentada en dispositivos móviles para facilitar y agilizar el proceso de diseño de interiores.

\subsection{Objetivos particulares}
\begin{itemize}
	\item Implementar una forma de visualización de muebles utilizando la realidad aumentada.
	\item Desarrollar una nueva propuesta de diseño de interiores para que los diseñadores de interiores puedan presentar a sus clientes.
	\item Poder almacenar las propuestas de diseño de interiores generadas.
	\item Desarrollar una forma para poder presentar análisis presupuestales.
	\item Poder agrupar las propuestas de diseño de interiores por cliente y almacenar información del mismo.
	\item Poder ajustar el costo final de los escenarios creados a través de la definición inicial del presupuesto del cliente.
\end{itemize}
	   \section{Justificación}
Cuando nos cambiamos de hogar, inevitablemente tenemos que afrontarnos con la tarea de decorar las habitaciones que hay en él. En éste punto, hacerlo no resulta tan complicado dado que partimos de una habitación vacía y ésta se convierte en un lienzo en blanco para nuestra imaginación. Al no haber objetos presentes, la percepción espacial de quien decora no se ve afectada, de tal forma que éste escenario facilita el diseño de interiores. Desafortunadamente no siempre tenemos la oportunidad de decorar una habitación cuando ésta se encuentra vacía, pues normalmente ya hay muebles y objetos decorativos en ella, entonces el proceso se resume a agregar nuevos objetos. Si nosotros escogemos un mueble que se ve agradable a simple vista, puede que, al momento de colocarlo en la habitación, no se vea tan bien y no se encuentre en armonía con los demás objetos decorativos. Incluso puede que el objeto ni siquiera quepa en el lugar donde se había planeado su posición y sea necesario reordenar la habitación, lo cual es cansado, dependiendo del peso y la posición de los muebles.\par 
Aunado a esto, puede llegar el punto donde quien decora la habitación, al final ya no desee el mueble, y realice un proceso de devolución de producto, si es que la tienda donde lo compró lo permite. Entonces la tienda pasa al domicilio donde se encuentre el producto para recogerlo o el usuario va a la tienda a entregarlo. De cualquier forma, se traduce en una pérdida económica y de tiempo.\par 
Todas éstas consecuencias se podrían evitar si realizamos un diseño de interiores efectivo, es decir, podamos saber si un mueble se va a ver bien en nuestra sala o comedor incluso antes de comprarlo. También facilitaría el proceso, saber las propiedades del producto, como pueden ser el peso y sus dimensiones exactas para determinar si se requiere un flete o no.\par 
Por lo anterior, el diseño de interiores  para las personas se vuelve complejo, tardado y tedioso, lo cual puede provocar pérdidas económicas y de tiempo por parte del cliente que compra un mueble y/o por parte de la tienda si se efectúa un proceso de devolución de producto dañando el prestigio de la tienda o sucursal asociada a la venta de estos muebles u objetos.

    \chapter{Marco Teórico }
	El presente trabajo pretende analizar y documentar el desarrollo de una aplicación móvil para el diseño de interiores, por ello las definiciones que a continuación se exponen son necesarias para entender objetivo y el funcionamiento del software.
	
	   \section{Diseño de interiores}
\subsection{Definicion}
El diseño de interiores es una profesión en la cual soluciones creativas y técnicas son aplicadas dentro de una estructura para lograr la construcción de un entorno interno determinado. Éstas soluciones son funcionales, mejoran la calidad de vida de los ocupantes y son aestéticamente atractivas. Los diseños deben apegarse al código y normas requeridos, y fomentar los principios de sustentabilidad ambiental definidos por el edificio o empresa. El objetivo del diseño de interiores es lograr una armonía en los espacios que habitamos y dar confort al usuario de dichos espacios\cite{B01}. \par
El diseño de interiores sigue una metodología sistemática y coordinada que incluye investigación, análisis e integración de conocimientos dentro de un proceso creativo. Dentro ésta metodología podemos ubicar distintos servicios o etapas, dependiendo de la complejidad del trabajo, en las cuales encontramos: definición de los requerimientos funcionales para los espacios de las habitaciones, planeación de espacios interiores, realización de planos de construcción, definición de especificaciones de ubicación, colores y acabados en piso, paredes, materiales, equipo, mobiliario y muebles, administración de contratos de fabricación o instalación, etc.\par
En Estados Unidos el diseño de interiores es la única rama del diseño que está sujeta a las regulaciones federales y la ley gubernamental\cite{B02}.
\subsection{Movimientos en el diseño de interiores}

\subsubsection{Feng shui}
"El Feng Shui es un arte utilizado actualmente para alcanzar la armonización de las energías en las casas y los lugares de trabajo, basado en principios milenarios de la sabiduría china"\cite{B26}. Surge de la conjunción de dos ideogramas chinos que significan "viento" y "agua", dos conceptos que para las tradiciones de la antigüedad se relacionaban con el flujo y la circulación de la energía vita. Mediante este arte, nos es posible conocer cuál es la perfecta ubicación para edificar una casa, el lugar ideal para colocar cada uno de los muebles, como así también la forma de revertir las energías adversas que puedan afectarnos. El Feng Shui estudia la relación del hombre con la naturaleza y brinda la oportunidad de vivir de acuerdo con los principios que la rigen, y de esta manera, aprovechar esas energías que fluyen por todas partes y pueden influir en nuestro bienestar general.

\subsubsection{Deconstructivismo}
El desconstructivismo es la “Arquitectura que busca llegar a nuevas formas de expresión al alejarse de las restricciones estructurales y jerarquías funcionales y temáticas, enfocado 
hacia diseños a menudo no rectangulares, fantásticos y aparentemente inconexos”\cite{B24}. Tal trabajo a menudo representa una aplicación de las teorías filosóficas  de Jacques Derrida en Francia, que trato de llegar a nuevas ideas en la literatura; esta filosofía se ha aplicado desde finales del siglo 20 a las estructuras arquitectónicas generalmente llamadas arquitecturas deconstructivistas. \par
La arquitectura deconstructivista surge en una exposición, titulada deconstructivist architecture, que Philip Johnson y Mark Wigley organizaron en el museo de Arte Moderno (MoMa) de Nueva York en 1988.

\subsubsection{Diseño Orgánico}
Arquitectura cuyo diseño se establece de acuerdo con los procesos de la naturaleza en lugar de basarse en un diseño ya impuesto. Es una filosofía de diseño propuesta por Frank Lloyd Wright (1867-1959) a comienzos del siglo 20 y en ella afirma que un edificio (y su apariencia) deben de seguir formas que estén en armonía con su entorno natural.\cite{B24}\par  
Los materiales utilizados en el exterior deben ser acoplarse  con la ubicación del edificio, relacionando así el edificio a su entorno. Por lo tanto, debe hacerse de baja altura, con techos que sobresalgan para proporcionar protección del sol en el verano y para proporcionar alguna protección contra la intemperie en invierno además se debe de hacer un máxima uso de la luz natural.

\subsection{Fundamentos del diseño de interiores}
El diseño de interiores se ve como una actividad que tiene un punto de inicio (cuando el diseñador y el cliente tienen el primer contacto) y otro al final cuando el proyecto se ha ejecutado.\par
Se debe tomar en cuenta que el diseño de interiores es maleable, es decir, que su realización no está sujeto estrictamente a una serie de reglas. En un caso se puede realizar un determinado proceso, y en otro se puede realizar otro proceso diferente. No existe una solución estandarizada para para todos los casos.\par
Lo más importante es definir el por qué estamos diseñando. Por ejemplo si se está diseñando un armario, se tiene qué saber cuál es el impulso para hacerlo. El diseñador se plantea algunas ideas sobre las funciones que tiene un armario, el uso de la madera, reciclada, el del plástico o el nuevo material, y con base a eso, define el objetivo de diseñar el armario.
Otro fundamento importante es la armonía que se busca. Un espacio interior no sólo debe verse bonito, y tener colores agradables a la vista. Cada elemento que compone un espacio debe relacionarse con los demás. Un sillón en una sala de espera debe relacionarse y tener alguna conexión con la mesa de centro. Esta relación puede ser la similitud del acabado de ambos, la ubicación de uno con respecto al otro, etc.
Una habitación debe seguir un esquema de colores bien definido. Dentro de estos esquemas tenemos el monocromático, complementario y análogo, y cada uno deriva del círculo cromático.\par

\textbf{Monocromático}.- Es una selección de colores que funcionen bien juntos. Esto es trabajar con un matiz, y la variación de tintes, tonos, y sombras.
\begin{figure}[h!]
	\centering
	\includegraphics[width=7cm]{imagenes/marcoteorico/disenointeriores/monocromatico.png}
	\caption{Esquema monocromático.\cite{B13}}
	\label{fig:monocromatico}
\end{figure}

\textbf{Complementario}.- Los colores que se encuentran en extremos opuestos del círculo cromático se consideran complementarios. Al combinar estos dos colores, se puede expresar contraste e interés. Son difíciles de usar en grandes cantidades, pero por su contraste son muy buenos para resaltar algo, como un llamado de atención.
\begin{figure}[h!]
	\centering
	\includegraphics[width=7cm]{imagenes/marcoteorico/disenointeriores/complementario.png}
	\caption{Esquema complementario.\cite{B13}}
	\label{fig:complementario}
\end{figure}

\textbf{Análogo}.- Los colores que se encuentran al lado en el círculo cromático, son agradables juntos. Son la combinación perfecta, ya que son perfectos para cualquier uso, incluso para resaltar y contrastar un elemento específico sin demasiada interrupción. Como regla general, se debe seleccionar un color dominante, un segundo color para sustentar, y un tercer color para acentuar.
\begin{figure}[h!]
	\centering
	\includegraphics[width=7cm]{imagenes/marcoteorico/disenointeriores/analogo.png}
	\caption{Esquema análogo.\cite{B13}}
	\label{fig:analogo}
\end{figure}


\par 

	   \section{Realidad Aumentada}

\subsection{Definición}
La realidad aumentada (AR) es una aproximación visual interactiva en tiempo real en la cual objetos virtuales son añadidos al entorno real, normalmente en la parte superior de un video, usando gráficos de computadora o móviles.\cite{B04} \par
Básicamente la realidad aumentada se refiere a que imágenes virtuales hechas por computadora sean mezcladas con la vista real para crear una viisión con elementos agregados. Su principal fundamento es mezclar la realidad con la realidad virtual. Aunque no sólo se trata de mezclar la realidad virtual, también pueden mezclarse elementos como audio, sensaciones, tacto, olores, y gusto los cuales son superpuestos sobre el mundo real para producir un \textbf{entorno de realidad aumentada}.\cite{B05}
Ésta tecnología no es nueva, tuvo sus orígenes en la década de los 90s\cite{B04}, y su primer aparición a nivel mundial fue en Octubre de 1998 en The FIrst International Workshop on AR (IWAR'98) en San Francisco\cite{B05}.\par

\subsection{Tipos de realidad aumentada}

\subsection{Impacto de la realidad aumentada}
Desde su aparición en los 90s, la realidad aumentada ha sido una tecnología de punta la cual sólo estaba disponible en algunos laboratorios y de forma remota al público en general
Más allá de ser una útil técnica de visualización, actualmente es usada en muchos sectores como la ingenería, la medicina, la robótica, la milicia, la educación, entretenimiento, etc.

\subsection{Aplicaciones de la realidad aumentada}
La realidad aumentada es usada para sistemas de visión incorporada o HUD (Head Up Display) y herramientas de información como Sekai Camera, el cual es un software para la cámara del teléfono usado para desplegar información de los objetos a los que la cámara es enfocada, por ejemplo edificiones. Actualmente sólo está disponible en Japón.
Posicionar objetos virtuales dentro de un entorno real ayuda a entender las relaciones espaciales y las dimensiones, requeridas por ejemplo en el diseño de interiores, la cual es una aplicacion prominente de la realidad aumentada. 
\subsection{Plataformas de realidad aumentada para dispositivos mófilves}

\subsubsection{Wikitude}
\subsubsection{ARKit}
\subsubsection{Vuforia}
\subsubsection{ARToolKit}
\subsubsection{ARCore}
ARCore es la plataforma de Google para construir experiencias de realidad aumentada. A través de diferentes APIs, ARCore le da la habilidad a un dispositivo móvil para percibir su entorno, entender el mundo e interactuar con la información. Algunas de nuestras APIs están disponibles a través de Android y iOS para crear experiencias de realidad aumentada compartidas.\par
ARCore tiene nueve pilares básicos que permiten integrar contenido virtual en el entorno real, por medio de la cámara de un dispositivo móvil:

\begin{itemize}
	\item \textbf{Rastreo de movimiento}.- Cuando el celular se mueve a través del mundo real, ARCore usa un proceso llamado \textbf{Odometría concurrente y mapeo} o simplemente \textbf{COM}, para entender en qué posición se encuentra con respecto al entorno. ARCore detecta visualmente distintas características en las imágenes capturadas por la cámara para definir puntos llamados \textbf{puntos característicos}, a partir de los cuales virtualmente genera una maya de puntos, y los usa para computar el cambio físico de su posición. Esto se combina con mediciones realizadas por el teléfono para determinar la posición y la orientación de la cámara relativa al mundo en tiempo real.
	
	\item \textbf{Entendimiento ambiental}.- ARCore trata de identificar una maya de puntos coincidentes dentro de una misma superficies, como una mesa, el suelo o un muro, y hace que cada una de estas superficies detectadas esté disponible como un \textbf{plano}. ARCore también es capaz de detectar los límites de cada plano y convertir esto en información útil para las aplicaciones, de tal forma que sea posible posicionar objetos virtualmente sobre cada uno de estos planos. Debido a que ARCore usa puntos característicos para detectar planos, superficies sin texturas como parades blancas pueden no ser detectadas apropiadamente.
	
	\item \textbf{Estimación de luz}.- ARCore puede detectas información sobre la iluminación del entorno y proveer a la cámara de dispositivo móvil de una correción de color y gamma, para lograr una imágen óptima. Esta información permite variar la iluminación de una habitación y ver cómo la iluminación también varía en los elementos virtuales, aumentando la sensación de realismo.
	
	\item \textbf{Interacción con el usuario}.- Una vez que se ha definido la maya de puntos, ARCore permite que el usuario interactúe con el entorno visualizado en la cámara del dispositivo. El usuario puede hacer tap, o realizar gestos con los dedos sobre la pantalla del móvil, y ARCore tiene la capacidad de interpretar y transportar éstas acciones al entorno virtual, por ejemplo, el usuario puede agregar un objeto en alguna superficie al tocar la pantalla del dispositivo con el dedo, tras esto se hace una cálculo de las coordenadas relativas X, Y del plano donde se va a agregar el objeto, y finalmente el objeto es mostrado a través de la cámara.
	
	\item \textbf{Puntos de orientación}.- Los puntos de orientación permiten colocar objetos virtuales en superficies inclinadas. Cuando se está detectando la maya de puntos y se detectan superficies cercanas con diferente ángulo, ARCore se encarga de calcular el ángulo de inclinación de tal superficie.
	
	\item \textbf{Anclas y rastreables}.- La posición y orientación puede cambiar conforme ARCore mejora su entendimiendo del entorno. Cuando se quiere posicionar un objeto virtual, se requiere definir una ancla para asegurar que su posición sea rastreada en tiempo real. Los planos y puntos son un tipo de objeto especial denominado \textbf{rastreable}. Como su nombre sugiere, son objetos que ARCore estará rastreando a lo largo del tiempo. Es posible anclar objetos a rastrables, de tal forma que estos objetos virtuales van a conservar su posición relativa en el mundo, no importando si la cámara se mueve e inclusive si estos salen del foco de visualización.
	
	\item \textbf{Imágenes aumentadas}.- Las imágenes aumentadas permiten desarrollar aplicaciones de realidad aumentada que puedan responder a imagenes en 2D específicas como paquetes o posters de películas. Los usuarios pueden tener experiencias de realidad aumentada cuando ellos enfocan la cámara del celular a uno de estos elementos, por ejemplo, al enfocar la cámara el póster de una película se puede hacer que un personaje aparezca dentro del entorno virtual, y desaparezca cuando la cámara deje de enfocar a la imagen.
	
	\item \textbf{Compartir}.- \textbf{Cloud Anchors API}) permite crear experiencias de realidad aumentada colaborativas con otras personas. Con ésta API, un dispositivo puede enviar un ancla y puntos característicos a a la nube. Éstas anclas pueden ser compartidas con otros usuarios ya sea de Android o iOS. Esto habilita a las apps para renderizar los mismos objetos 3D asociados a tales anclas, permitiendo a los usuarios tener la misma experiencia de realidad aumentada de forma simultánea
	
\end{itemize}

\noindent
Dado que ARCore es una tecnología nueva (su primer publicación oficial fue en febrero de 2018 y su primer versión estable fue en agosto de 2018) la lista de dispositivos compatibles con ella no es tan amplia. En la \textit{Tabla 2} podemos observar una lista con los dispositivos que actualmente soportan ARCore.

% Please add the following required packages to your document preamble:
% \usepackage{multirow}
\begin{table}[]
	\begin{tabular}{|c|l|}
		\hline
		\textbf{Marca}              & \multicolumn{1}{c|}{\textbf{Modelo}}               \\ \hline
		\multirow{2}{*}{ASUS}       & Zenfone AR                                         \\ \cline{2-2} 
		& Zenfone ARES                                       \\ \hline
		\multirow{4}{*}{Google}     & Nexus 5X                                           \\ \cline{2-2} 
		& Nexus 6P                                           \\ \cline{2-2} 
		& Pixel, Pixel XL                                    \\ \cline{2-2} 
		& PIxel 2, Pixel 2 XL                                \\ \hline
		\multirow{5}{*}{HDM Global} & Nokia 6                                            \\ \cline{2-2} 
		& Nokia 6.1 Plus                                     \\ \cline{2-2} 
		& Nokia 7 Plus                                       \\ \cline{2-2} 
		& Nokia 8                                            \\ \cline{2-2} 
		& Nokia 8 Sirocco                                    \\ \hline
		\multirow{4}{*}{Huawei}     & Honor 10                                           \\ \cline{2-2} 
		& nova 3, nova 3i                                    \\ \cline{2-2} 
		& P20, P20 Pro                                       \\ \cline{2-2} 
		& Porsche Design Mate RS                             \\ \hline
		\multirow{4}{*}{LG}         & G6                                                 \\ \cline{2-2} 
		& G7 ThiQ                                            \\ \cline{2-2} 
		& V30, V30+, V30+ JOJO                               \\ \cline{2-2} 
		& V35 ThinQ                                          \\ \hline
		\multirow{7}{*}{Motorola}   & Moto GS5 Plus                                      \\ \cline{2-2} 
		& Moto G6                                            \\ \cline{2-2} 
		& Moto G6 Plus                                       \\ \cline{2-2} 
		& Moto X4                                            \\ \cline{2-2} 
		& Moto Z2 Force                                      \\ \cline{2-2} 
		& Moto Z3                                            \\ \cline{2-2} 
		& Moto Z3 Play                                       \\ \hline
		\multirow{4}{*}{OnePlus}    & OnePlus 3T                                         \\ \cline{2-2} 
		& OnePlus 5                                          \\ \cline{2-2} 
		& OnePlus 5T                                         \\ \cline{2-2} 
		& OnePlus 6                                          \\ \hline
		\multirow{10}{*}{Samsung}   & Galaxy A5                                          \\ \cline{2-2} 
		& Galaxy A6                                          \\ \cline{2-2} 
		& Galaxy A7                                          \\ \cline{2-2} 
		& Galaxy A8, Galaxy A8+                              \\ \cline{2-2} 
		& Galaxy Note8                                       \\ \cline{2-2} 
		& Galaxy Note9                                       \\ \cline{2-2} 
		& Galaxy S7, Galaxy S7 edge                          \\ \cline{2-2} 
		& Galaxy S8, Galaxy S8+                              \\ \cline{2-2} 
		& Galaxy S9, Galaxy S9+                              \\ \cline{2-2} 
		& Galaxy Tab S4                                      \\ \hline
		\multirow{3}{*}{Sony}       & Xperia XZ Premium                                  \\ \cline{2-2} 
		& Xperia XZ1, Xperia XZ1 Compact                     \\ \cline{2-2} 
		& Xperia XZ2, Xperia XZ2 Compact, Xperia XZ2 Premium \\ \hline
	\end{tabular}

\captionsetup{justification=centering}
\caption*{Tabla 2. Listado de dispositivos compatibles con ARCore}
\end{table}


----------------------------------------------

%La mayoría de las veces asociamos los términos de realidad aumentada con la realidad virtual como si fueran lo mismo, sin embargo existen grandes motivos para detallar sus diferencias de software, hardware y de la interacción con los usuarios. \\.\par 

%La realidad aumentada (AR) nos permite sobreponer objetos en tercera dimensión sobre una imagen en tiempo real, obteniendo una mezcla de lo real con lo virtual, mejorando la percepción del mundo real de un usuario. Estas imagenes en tiempo real las podemos obtener con las camaras de smartphones, tablets, smart glasses y computadoras. [1] \\. \par

%Por otro lado la realidad virtual (VR) se define como un sistema informático que genera en tiempo real representaciones de una realidad es decir, un mundo virtual donde puede llegar a existir una interacción con el usuario. Muchas de estas simulaciones requieren de gafas de VR compatibles con smartphones o gafas especiales como el Gear VR desarrollado por Samsumg en colaboración con Oculus VR. [2][3] \\. \par

%A diferencia de la VR, la AR es una tecnología que complementa  la percepción  e interacción  con el mundo real y permite al usuario estar en un entorno aumentado con información generada por computadora. [1] \\. \par

%Actualmente, existen dos grandes representantes en estas tecnologías. Tal vez por criterios de marketing diríamos que Google Glass es el representante de la realidad aumentada, mientras que el Oculus Rift de Facebook sería el representante de la realidad virtual. [4]

    %%%%%%%   ANALISIS   %%%%%%%
	\chapter{Analisis } %CAP\'ITULO 3
En la siguiente sección se detalla el análisis los requerimientos y diagramas de la aplicación móvil. Además, se detallan los diferentes escenarios que se presentan en la aplicación móvil. 
	
	   \section{Descripción general de la aplicación}
Este sistema móvil sera capaz de  brindarle al usuario final una herramienta que le permita observar muebles virtuales y al mismo tiempo de percibir la realidad obtenida por una cámara,\par

Se observará un video en tiempo real de la información virtual sobrepuesta dentro del entorno real por medio de la cámara. Las idea es darle una perspectiva más exacta en forma, color, dimensión nuestro mueble u objeto de agrado. Los resultados de obtener esta realidad mezclada es generar acierto, certidumbre y seguridad a las necesidades de un usuario con pretensiones de adquirirlo mediante una compra. Esta aplicación será enfocada al e-Commerce y los compradores potenciales serán mueblerías y establecimientos de venta de artículos de decoración y hogar. Para concluir enlistaremos las principales caracterisitcas que tendra este TT.\par
\begin{enumerate}[1.]
	\item La aplicación debe ser un sistema móvil capaz de comunicarse con los servicios de AWS (Amazon Web Services). Esta plataforma nos ofrece todo lo necesario para la interacción y persistencia de la información. AWS será el encargado de alojar la información dentro de una instancia de RDS que tendrá un motor MySQL donde se encontraran los usuarios y la información de los muebles. Por otro lado AWS nos ofrece S3 para el alojamiento de las imagenes por medio de S3 Bucket. 
	\item Los datos registrados por cada usuario dentro de la aplicación será un nombre, correo y contraseña.\par
	\item Cada imagen será alojada en formato base64. Por otro lado el tamaño en bytes de cada objeto dependerá de la complejidad, textura, colores, dimensiones y detalles que influyan significativamente en el proceso de renderización en un dispositivo anfitrión.\par
	\item Los usuarios que interactúen con la aplicación deberá contar con los permisos necesarios para gestionar  la interacción con hardware o software disponible en el dispositívo.\par	
\end{enumerate}

	   	   \section{Actores}

\subsection{Usuario final}
\textbf{Nombre del actor:} \textit{Usuario}\par
\textbf{Definición:} Pueden ser dos diferentes: un diseñador de interiores o un habitante común de un inmueble. Ambos tendrán las mismas funcionalidades disponibles dentro de la aplicación. El habitante común del inmueble puede iniciar sesión como huésped (no necesita crear cuenta) para poder usar la funcionalidad de realidad aumentada aunque no podrá guardar escenarios ni proyectos, mientras que el diseñador de interiores podrá usar todas las funcionalidades de la aplicación.
	   \section{Requisítos Funcionales}
En esta seccion se detallaran los requisitos necesarios para el funcionamiento de la aplicación.\par
\vspace{5mm}
\begin{enumerate}[1.]
\item El usuario debe validarse por medio de un login para poder realizar cualquier función en la aplicación 
\item El usuario podra seleccionar un mueble desde un catálogo. 
\item El usuario podrá modificar el color del mueble a visualizar. 
\item Implementación de la aplicación móvil 
\item El usuario obtendrá una imagen del mueble sobre la realidad que observa.
\end{enumerate}
   	   \section{Casos de uso}
El la siguiente seccion se define y describen los esenarios entre los actores y ARF describiendo las entidades de negocio y su funcionalidad dentro de la aplicación, así mismo su interacción con los actores interesados con la aplicación.\par
\begin{figure}[h!]
	\centering
	\includegraphics[width=9cm,height=11cm]{imagenes/analisis/casosDeUso.jpg}
	\caption{Casos de uso del usuario.}
	\label{fig:analogo}
\end{figure}  
\newpage

\subsection{Actores}
\textbf{Nombre del actor:} \textit{Cliente} \textbf{(Usuario)}\par
\textbf{Definición :} Usuario final y el principal interesado en usar nuestra aplicación. Tendrá los permisos de un usuario estándar, este actor sera el único considerado hasta este momento ya que en iteraciones posteriores se contempla el aumento de usuarios. A continuación se describen las funciones donde interviene la aplicación este usuario.

\subsection{Login}
\begin{figure}[h!]
	\centering
	\includegraphics[width=9cm,height=5cm]{imagenes/analisis/login.jpg}
	\caption{CU01 Login.}
	\label{fig:analogo}
\end{figure}  

\subsection{Registrar usuario} 
\vspace{5mm} 	
\begin{figure}[h!]
	\centering
	\includegraphics[width=10cm,height=3cm]{imagenes/analisis/registro.jpg}
	\caption{CU02 Registrar un usuario.}
	\label{fig:analogo}
\end{figure} 
\newpage
\subsection{Seleccionar mueble} 
\vspace{5mm}
\begin{figure}[h!]
	\centering
	\includegraphics[width=9cm,height=4cm]{imagenes/analisis/seleccionarMueble.jpg}
	\caption{CU3 Seleccionar mueble.}
	\label{fig:analogo}
\end{figure}

\subsection{Gestión de escenario} 
\vspace{5mm}
\begin{figure}[h!]
	\centering
	\includegraphics[width=12cm,height=6cm]{imagenes/analisis/gestionEscenarios.jpg}
	\caption{CU04 Gestionar escenario.}
	\label{fig:analogo}
\end{figure}

   	   \newpage
\section{Diagramas de secuencia}
A continuación se describe la interacción de nuestras entidades de negocio y el ciclo de vida que tendrán en la aplicación. También se detalla la interacción, mensajes y la lógica implementada en cada uno de los diferentes escenarios.\par

\subsection{DS1. Inicio de sesión}
El inicio de sesión consiste en la captura de usuario y contraseña de una cuenta para tener acceso a sus datos. Estos dos datos se envían en formato JSON a AWS API Gateway que sirve de puerta de enlace para poder establecer una comunicación con la AWS Lambda \textit{ARFLogin} la cual se encarga de realizar una consulta en una base de datos en MySQL montada sobre una instancia de AWS RDS para verificar que los datos correspondan a una cuenta registrada. El resultado de esta consulta, ya sea exitosa o fallida, se devuelve a ARF en formato JSON a través de AWS API Gateway. Esta respuesta tiene dos parámetros que indican el estatus de la operación: \textbf{code} y \textbf{message}

\begin{figure}[h!]
	\centering
	\includegraphics[width=14cm,height=14cm]{imagenes/analisis/ds/dsinicio_sesion.jpg}
	\caption{DS1. Inicio de sesión.}
	\label{fig:dsiniciosesion}
\end{figure}

\clearpage

\subsection{DS2. Registro de cuenta}
El registro de cuenta consiste en la captura de nombre, email y contraseña. La contraseña se deberá repetir. Esos datos se envían en formato JSON a AWS API Gateway la cual se encarga de ejecutar la AWS Lambda \textit{ARFRegisterAccount}. Esta función se encarga primero de consultar el correo para verificar que se cumpla la regla de negocio \textbf{BR1}. Si la regla se cumple, entonces el usuario es registrado por esta función. Ya sea que la inserción sea realizado con éxito o no, la función construye nu JSON de respuesta que contiene el status de la operación, y esta respuesta es manipulada por la aplicación para mostrar en pantalla un error, o redirigir el usuario a la pantalla de Inicio de sesión.

\begin{figure}[h!]
	\centering
	\includegraphics[width=14cm,height=16cm]{imagenes/analisis/ds/RegCuenta.jpg}
	\caption{DS2. Registro de cuenta.}
	\label{fig:dsregcuenta}
\end{figure}
\clearpage

\subsection{DS3. Recuperación de cuenta}
El proceso de recuperación se divide en tres pasos:
\begin{itemize}
	\item \textbf{Envío de código:} El usuario introduce el email de su cuenta, la aplicación se comunica con la AWS Lambda \textit{ARFRecoverAccount} a través de la AWS API Gateway con mensajes en formato JSON. La Lambda se encarga de verificar que el email pertenezca a una cuenta registrada. De ser así, se envía un email a tal cuenta con un código de verificación generado para esa cuenta, en otro caso se muestra un mensaje del error. La función responde en JSON indicando el estatus de esta operación. De ser exitosa, la pantalla de \textit{Verificación de código}.
	\item \textbf{Verificación de código:} El usuario ingresa el código que fue enviado a su correo. De nuevo se realiza la comunicación a la AWS Lambda \textit{ARFRecoverAccount} via AWS API Gateway. La función se encarga de validar que el código ingresado por el usuario corresponda al que se generó y envió previamente. La Lambda responde con el status de esta validación. Si la validación es correcta, la aplicación muestra la pantalla de \textit{Reestablecimiento de contraseña}, en caso contrario muestra un mensaje describiendo el error.
	\item \textbf{Reestablecimiento de contraseña:} El usuario captura la nueva contraseña que desea que tenga su cuenta. De nuevo se realiza la comunicación a la AWS Lambda \textit{ARFRecoverAccount} via AWS API Gateway. La función se encarga de actualizar en la base de datos la contraseña del usuario, y responde con el status de esta operación. Finalmente el usuario es redireccionado al Inicio de sesión para poder acceder a su cuenta con la nueva contraseña.
\end{itemize}

\begin{figure}[h!]
	\centering
	\includegraphics[width=13cm,height=22cm]{imagenes/analisis/ds/RecCuenta.jpg}
	\caption{DS3. Recuperación de cuenta.}
	\label{fig:dsreccuenta}
\end{figure}
\clearpage

\subsection{DS4. Gestión de proyectos}
Cuando se desea crear un escenario debe crearse antes un proyecto que lo contenga, para poder tener una gestión de proyectos y escenarios. en este escenario se muestra el proceso de creación de un proyecto por parte del usuario: El usuario llena un formulario donde le pide datos del cliente del proyecto (nombre, apellido, teléfono y correo). Tras llenarlos y seleccionar \textit{Crear proyecto}, la aplicación envía estos datos a la AWS Lambda \textit{ARFProject} en formato JSON a través de la AWS API Gateway, la cual se encarga de hacer la inserción del proyecto en la base da datos con base en los datos recibidos y responde con el status de tal operación. La aplicación recibe esta respuesta y muestra un mensaje con error en caso de haberlo o redirige al usuario a la pantalla de \textit{Visualización de escenarios} donde ya se muestra el proyecto que acaba de generar.
%\begin{itemize}
%	\item \textbf{Creación de proyecto:} El usuario llena un formulario donde le pide datos del cliente del proyecto (nombre, apellido y  teléfono). Tras llenarlos y seleccionar \textit{Crear proyecto}, la aplicación envía estos datos a la AWS Lambda \textit{ARFProject} en formato JSON a través de la AWS API Gateway, la cual se encarga de hacer la inserción del proyecto en la base da datos con base en los datos recibidos y responde con el status de tal operación. La aplicación recibe esta respuesta y muestra un mensaje con error en caso de haberlo o redirige al usuario a la pantalla de \textit{Visualización de escenarios} donde ya se muestra el proyecto que acaba de generar.
%	\item \textbf{Actualización del proyecto: } Dentro de un proyecto el usuario modifca los campos que desee actualizar y selecciona \textit{Actualizar}. Estos datos se envían en formato JSON a la AWS Lambda \textit{ARFProject} via AWS API Gateway, la cual se encarga de actualizar en la base de datos el proyecto correspondiente de acuerdo a la información recibida y responde con el status de tal operación. La aplicación recibe esta respuesta y muestra un mensaje con error en caso de haberlo o redirige al usuario a la pantalla de \textit{Visualización de proyectos}.
%	\item \textbf{Eliminación de proyecto:} El usuario selecciona el icono de eliminación de alguno de los proyectos listados. La aplicación se comunica a la AWS Lambda \textit{ARFProject} via AWS API Gateway, la cual se encarga de hacer la eliminación del proyecto solicitado en la base de datos y responde con el status de tal operación. La aplicación recibe esta respuesta y muestra un mensaje con error en caso de haberlo o redirige al usuario a la pantalla de \textit{Visualización de proyectos}.
%	\item \textbf{Obtención de proyecto:} Cuando el usuario selecciona un proyecto para visualizar su información, se hace una petición a la AWS Lambda \textit{ARFProject} para obtener la información completa del proyecto en cuestión desde la base de datos y la Lambda responde con el proyecto solicitado. Finalmente la aplicación muestra los datos recibidos a través de un formulario que permitirá actualizar el proyecto.
%	\item \textbf{Obtención de proyectos de usuario:} Cuando un usuario entra a la pantalla de \textit{Visualización de proyectos} se hace una petición a la AWS Lambda \textit{ARFProject} via AWS API Gateway, la función se encarga de obtener de la base de datos todos los proyectos que pertenezcan al usuario que realiza la solicitud y responde con esta información. Finalmente la aplicación recibe esta respuesta y despliega un listado de los proyectos obtenidos.
%\end{itemize}

\begin{figure}[h!]
	\centering
	\includegraphics[width=14cm,height=6cm]{imagenes/analisis/ds/CreateProject.jpg}
	\caption{DS4. Gestión de proyectos.}
	\label{fig:dsreccuenta}
\end{figure}
\clearpage


\subsection{DS5. Gestión de escenarios}
Aquí se presentan cinco escenarios: creación de escenario, actualización de escenario, eliminación de escenario, obtención de escenario, y obtención de escenarios de proyecto.
\begin{itemize}
	\item \textbf{Creación de escenario:} Tras finalizar la creación del entorno de realidad aumentada y haber tomado las fotografías deseadas, el usuario confirma la creación de un proyecto, entonces la aplicación se comunica con la AWS Lambda \textit{ARFScenario} via AWS API Gateway, a la cual le envía la información general del escenario y las fotografías y videos realizados. Esta función se encarga de almacenar en AWS S3 los archivos recibidos y generar un registro de escenario en la base de datos que también asocie estos recursos almacenados a través de su URL pública de descarga. Tras esto, la función responde con el registro del escenario creado. La aplicación recibe esta respuesta y regresa a la pantalla de \textit{Visualización de escenarios} y muestra un mensaje de error en caso de haberlo.
	\item \textbf{Actualización de escenario: } Dentro de un escenario el usuario modifca los campos que desee actualizar y selecciona \textit{Actualizar}. Estos datos se envían en formato JSON a la AWS Lambda \textit{ARFScenario} via AWS API Gateway, la cual se encarga de actualizar en la base de datos el escenario correspondiente de acuerdo a la información recibida y responde con el status de tal operación. La aplicación recibe esta respuesta y muestra un mensaje con error en caso de haberlo o redirige al usuario a la pantalla de \textit{Visualización de escenarios}.
	\item \textbf{Eliminación de escenario:} El usuario selecciona el icono de eliminación de alguno de los escenario listados. La aplicación se comunica a la AWS Lambda \textit{ARFScenario} via AWS API Gateway, la cual se encarga de hacer la eliminación del escenario solicitado en la base de datos y responde con el status de tal operación. La aplicación recibe esta respuesta y muestra un mensaje con error en caso de haberlo o redirige al usuario a la pantalla de \textit{Visualización de escenarios}.
	\item \textbf{Obtención de escenario:} Cuando el usuario selecciona un escenario para visualizar su información, se hace una petición a la AWS Lambda \textit{ARFScenario} para obtener la información completa del escenario en cuestión desde la base de datos y la Lambda responde con el escenario solicitado. Finalmente la aplicación muestra los datos recibidos a través de un formulario que permitirá actualizar el escenario.
	\item \textbf{Obtención de escenarios de usuario:} Cuando un usuario entra a la pantalla de \textit{Visualización de escenarios} se hace una petición a la AWS Lambda \textit{ARFScenario} via AWS API Gateway, la función se encarga de obtener de la base de datos todos los escenarios que pertenezcan al usuario que realiza la solicitud y responde con esta información. Finalmente la aplicación recibe esta respuesta y despliega un listado de los escenarios obtenidos.
\end{itemize}

\begin{figure}[h!]
	\centering
	\includegraphics[width=14cm,height=20cm]{imagenes/analisis/ds/CrearEscenario.jpg}
	\caption{DS5. Gestión de escenarios.}
	\label{fig:dsreccuenta}
\end{figure}
\clearpage

	%%%%%%%   DESARROLLO   %%%%%%%
	\chapter{Desarrollo}

En este capítulo se describirá el proceso de desarrollo del trabajo terminal de acuerdo a la metodología Mobile-D, de tal forma que estará divido en capítulos según las iteraciones que se vayan realizando.\par
		\section{Exploración}
\subsection{Establecimiento de interesados}
\subsection{Alcance}
\subsection{Establecimiento de proyectos}
\subsection{Pruebas de contexto}
Se realizaron pruebas de contexto definiendo las siguientes especificaciones:
\begin{itemize}
	\item Nombre del dispositivo en el que se realizó
	\item Fecha de prueba
	\item Versión de Sceneform SDK usada
	\item Versión de ARCore SDK
	\item Versión de Android
\end{itemize}
Por otro lado cada prueba se realizó tomando en cuenta las siguientes características:
\begin{itemize}
	\item \textbf{Posición Cardinal}.- Define si un objeto virtual puede visualizarse desde los cuatro puntos cardinales (norte, sur, este, oeste) y desde la parte superior del mismo, cuando la cámara gira alrededor del objeto mientras lo enfoca.
	\item \textbf{Tamaño relativo}.- Define si un objeto cambia su tamaño en el entorno virtual dependiendo de la distancia a la que se acerque o se aleje la cámara. Cuando la cámara se acerca, el objeto deberá aumentar su tamaño, y viceversa, como si se tratase de un objeto real.
	\item \textbf{Luminosidad}.- Define el grado de oscuridad de un objeto a determinada luz, es decir, cuando la luz en el ambiente real es alta, el objeto se verá iluminado, en caso contrario cuando haya escasa luz, el objeto se oscurecerá.
	\item \textbf{Superficie}.- Describe en qué superficies el objeto virtual es puesto, si fue posible su superposición en este material y qué comportamiento tiene en cada una de estas.
	\item \textbf{Memoria de objetos}.- Define si los objetos virtuales se conservan en la memoria cuando la cámara pierde su enfoque en ellos y los vuelve a enfocar. También describe si los objetos conservaron su posición tras el re-enfoque.
	\item \textbf{Capacidad máxima de objetos}.- Define el número de objetos virtuales que pueden ser mostrados en escena sin que el rendimiento de la aplicación caiga considerablemente.
	\item \textbf{Distancia}.- Define la distancia a la que se encuentra la cámara de un objeto virtual sin que éste desaparezca o sin que su resolución baje considerablemente.
\end{itemize}
\noindent

\subsubsection{MOTO G6 XT1925}
\begin{table}[!h]
	\centering
	\begin{tabular}{|c|c|}
		\hline
		\multicolumn{2}{|c|}{Especificaciones de prueba}   \\ \hline
		\textbf{DISPOSITIVO}              & Moto G6 XT1925 \\ \hline
		\textbf{FECHA}                    & 2018/08/25     \\ \hline
		\textbf{VERSIÓN DE SCENEFORM SDK} & V1.4.0         \\ \hline
		\textbf{VERSIÓN DE ARCORE SDK}    & V1.4.0         \\ \hline
		\textbf{VERSIÓN DE ANDROID}       & V8.0.0 (Oreo)  \\ \hline
	\end{tabular}
	\captionsetup{justification=centering}
	\caption{Especificaciones de prueba en Moto G6}
\end{table}

\textbf{Posición cardinal} \par
El objeto virtual se pudo apreciar con claridad desde los cuatro puntos cardinales y la vista superior. El ángulo de visualización del objeto al mover la cámara cambiaba a la perfección, dando una buena percepción de realismo.

%%IMAGENES DE PUNTOS CARDINALES


\textbf{Tamaño relativo} \par
Al acercar o alejar la cámara el objeto virtual variaba su tamaño de forma adecuada, como si el objeto realmente estuviera en la posición donde fue superpuesto.

%%IMAGENES DE TAAAÑO RELATIVO

\textbf{Luminosidad} \par
Al poner el objeto virtual en entornos con diferente cantidad de luz, la cantidad de luz en el objeto virtual también variaba. En un entorno con ausencia casi total de luz el objeto apenas era perceptible, mientras que en un entorno con bastante luz, el objeto se veía altamente iluminado.

%%IMAGENES DE LUMINOSIDAD

\textbf{Superficie} \par
Se probó posicionar el objeto virtual en cuatro superficies: concreto gris, concreto blanco, azulejo y vidrio.\par
Concreto gris.- El objeto se pudo posicionar a la perfección.\par
Concreto blanca.- El objeto no se pudo posicionar. La maya de puntos ni si quiera era detectada en ésta superficie debido a la ausencia de texturas.\par
Azulejo.- El objeto se pudo posicionar a la perfección.\par
Vidrio.- El objeto no pudo ser posicionado en ésta superficie debido a las propiedades reflejantes que posee.\par

%%IMAGENES DE SUPERFICIES

\textbf{Memoria de objetos} \par
Tras perder el enfoque de la cámara, al volverlo a tener, todos los objetos virtuales se volvieron a mostrar en el entorno virtual en la misma posición en la que habían sido puestos.

\textbf{Capacidad máxima de objetos} \par
Se colocaron 100 objetos virtuales en escena sin que la aplicación perdiera rendimiento. Todo funcionaba con total fluidez.

%%IMAGENES DE 100 OBJETOS

\textbf{Distancia} \par
Se colocó un objeto, después la cámara fue alejada hasta una distancia de \textbf{11.22m.} A esa distancia los objetos comenzaron a verse pixeleados, además comenzaron a desaparecer y reaparecer de forma intermitente.

%%IMAGENES DE DISTANCIA MÁXIMA




		\subsection{Inicialización}
En ésta fase se describirá la linea base sobre la que se desarrollará el proyecto, esto significa, definir las herramientas y dispositivos con los que se realizará el desarrollo, qué tecnologías vamos a usar y el planteamiento general del sistema.\par
\noindent
\\
\textbf{Herramientas y Tecnologías}:
	\begin{itemize}
			\item Amazon Web Services
			\item Android 8 Oreo y Android 7 Nougat
			\item Android Studio 2.19
			\item Arcore 1.5
			\item Blender 2.79
			\item Debian  9.5
			\item Git
			\item GitHub
			\item Java 7
	  		\item JSON
			\item Windows 10
			\item XML		
	\end{itemize}
	\noindent
\textbf{Dispositivos}:
\begin{itemize}
	\item Laptop Acer, procesador Intel Core i3 6ta generación, 4GB de RAM, 1TB en disco duro, sistema operativo Debian  9.5
	\item Laptop HP, procesador Intel Core i5 6ta generación, 4GB de RAM, 1TB en disco duro, sistema operativo Windows 10
	\item Laptop Toshiba, procesador Intel Core i3 5ta generación, 4GB de RAM, 1TB en disco duro, sistema operativo Debian 9.5
	\item Moto G6
	\item Moto G6 plus		
\end{itemize}
\noindent
\textbf{Arquitectura}:
El backend de la aplicación se encontrará sobre la infraestructura de Amazon Web Services (AWS), debido a la alta escalabilidad que proporciona. La arquitectura contendrá un cluster RDS con MySQL, cinco Lambdas, un Bucket de S3 y una API Gateway (véase Figura 3.15).\par
Cada Lambda estará enfocada a una funcionalidad principal de la aplicación:\par
\begin{itemize}
	\item\textbf{Login}.- Encargada de toda la lógica y la seguridad informática relacionada con la autenticación en la aplicación.
	\item\textbf{Registrar cuenta}.- Permitirá registrar una nueva cuenta en el sistema, misma que permitirá guardar escenarios.
	\item\textbf{Recuperar cuenta}.- En caso de que se olvide la contraseña de la cuenta, ésta Lambda contendrá la lógica para recuperar la cuenta
	\item\textbf{Guardar escenario}.- Permitirá almacenar las imágenes que sean enviadas. Estas imágenes serán almacenadas en un bucket de S3 y asociadas a un escenario. Esta información será almacenada en la base de datos de MySQL que está montada sobre el cluster de RDS.
	\item\textbf{Ver escenario}.- Ësta Lambda permitirá recuperar la información e imágenes de un escenario especificado, con el fin de poder visualizarlo desde la ubicación.
\end{itemize}
\noindent
La aplicación podrá comunicarse a toda la infraestructura de AWS por medio de la API Gateway, que sirve como puerta de enlace tanto para entrar como para salir de AWS. Ésta comunicación se realizará a través del protocolo HTTPS, dada la facilidad de uso que proporciona, además de la capa de seguridad SSL que ya proporciona AWS durante la comunicación con la API Gateway. En la figura 4.25 puede observarse un diagrama de arquitectura del sistema.
\begin{figure}[H]
	\centering
	\includegraphics[width=15cm,height=15cm]{imagenes/desarrollo/arquitectura/ArchitecturaBackend.png}
	\caption{Arquitectura de Backend de ARF.}
	\label{fig:arqbackend}
\end{figure}
\par
Por otro lado se requerirá una base de datos que pueda almacenar los usuarios registrados y los escenarios que estos vayan creando. Para el tipo y cantidad de información que se requiere almacenar, una base de datos relacional cumple a ésta necesidad. En la figura  3.16 describe el diseño de la base de datos que se va a usar en ARF, misma que se encuentra en el clúster Amazon RDS mostrado en la figura 4.26.
\begin{figure}[H]
	\centering
	\includegraphics[width=10cm,height=7cm]{imagenes/desarrollo/arquitectura/ERD.png}
	\caption{Modelo Entidad-Relación usado para la base de datos de ARF.}
	\label{fig:arqbackend}
\end{figure}

%\textbf{Hola mundo! ARcore}

%Los desarrolladores de ARcore nos proporcionan un quickstart el cual utilizamos para iniciar la curva de aprendizaje de esta herramienta. Este quickstart únicamente nos proporciona una clase principal, la cual contiene la lógica necesaria para mostrar un simple renderizado. 


%\begin{figure}[H]
%	\centering
%	\includegraphics[width=5cm]{imagenes/iteraciones/AR1.png}
%	\caption{Hola mundo de ARcore}
%	\label{fig:HMARcore}
%\end{figure}
		\section{Iteración I}
\subsection{Resumen}
\subsection{Diagramación}
\subsection{Desarrollo}


		\section{Iteración II}
\subsection{Resumen}
En esta iteración se modeló un mueble y se implementó para visualizarse en la aplicación.
\subsection{Desarrollo}
Ya entendido el funcionamiento de ARcore y como agregar nuevos modelos, en esta iteración se realizó el modelado de un objeto de mayor complejidad: una mesa metálica con algunos detalles de acabado. Al igual que en la iteración pasada se obtuvieron dos tipos de archivos y se renderizaron para mostrarse en la aplicación. El resultado de ésta iteración es el equivalente al entregable de TT I que propusimos en el protocolo, en donde planeamos que al final de la iteración IV ibamos a entregar una aplicación que a través de realidad aumentada pudiera mostrar un simple mueble sin cambios de color o textura.
En la figura 4.28 se puede observar el nuevo modelo implementado.
\begin{figure}[H]
	\centering
	\includegraphics[width=8cm,height=14cm,angle=90]{imagenes/iteraciones/AR3.png}
	\caption{Mueble modelado en 3D}
	\label{fig:analogo}
\end{figure} 

		\section{Iteración III}
\subsection{Resumen}
En ésta iteración se realizó el frontend del login de ARF así como el backend para los módulos "LOGIN", "REGISTRAR CUENTA" y "GUARDAR ESCENARIO" de acuerdo a la arquitectura descrita en la figura 4.25. \par

%\subsection{Diagramación}
\subsection{Desarrollo}
El desarrollo de los módulos de backend mencionados se realizó sobre Lambdas de AWS. Cada módulo tiene su propia función. El objetivo de hacerlo de esta forma es brindarle escalabilidad al sistema pues el desarrollo se logra de forma modular.\par
Cada uno de estos módulos es accesible a través del protocolo HTTP gracias a la API Gateway. La API Gateway es un módulo de Amazon Web Services que permite que los recursos de AWS sean accesibles desde el exterior, en éste caso, permite que la aplicación en Android pueda acceder a éstas funciones con una simple petición.\par
Las Lambdas y la API Gateway están configuradas para que a través del método POST y el formato JSON la aplicación pueda comunicarse y usar las funciones almacenadas en la nube.\par
El desarrollo de cada función en la nube se realizó con NodeJS v8.10.0. A continuación se describe cada módulo. \par
\subsubsection{Login}
Se encarga de la funcionalidad del login de la aplicación. En la figura 4.29 se puede observar el diagrama del proceso de login.\par
\begin{figure}[h!]
	\centering
	\includegraphics[width=15cm,height=9cm]{imagenes/desarrollo/diagramas/BPMN_LOGIN.png}
	\caption{Diagrama de proceso de Login.}
	\label{fig:loginsuccess}
\end{figure}
Ésta función recibe dos parámetros: usuario y contraseña. Posteriormente inicia un primer proceso de validación, el cual consiste en la obtención del hash criptográfico de la contraseña a través de SHA-512. Ese hash junto con el usuario es buscado en una base de datos relacional donde se encuentran los usuarios, esta base de datos está almacenada en Amazon RDS, que tiene un motor MySQL 5.5. Si encuentra alguna coincidencia entonces significa que el usuario existe.\par Así, la función regresa en formato JSON el usuario junto a su hash para que la aplicación realice una segunda etapa de verificación (ver figura 4.30). \par
\begin{figure}[h!]
	\centering
	\includegraphics[width=15cm,height=3.5cm]{imagenes/desarrollo/arquitectura/LOGIN_SUCCESS.png}
	\caption{Flujo de información de login exitoso.}
	\label{fig:loginsuccess}
\end{figure}

Por otro lado si los datos ingresados son incorrectos, la función responde con un código de error (-1000), que es el asociado a datos de login incorrectos (ver figura 4.31). 
\begin{figure}[h!]
	\centering
	\includegraphics[width=15cm,height=3.5cm]{imagenes/desarrollo/arquitectura/LOGIN_FAIL.png}
	\caption{Flujo de información de login fallido.}
	\label{fig:loginfail}
\end{figure}

\subsubsection{Registro}
Ésta función se encarga de registrar un nuevo usuario en la base de datos para que posteriormente pueda iniciar sesión en la aplicación. El modelado de este proceso se puede apreciar en la figura 4.32. \par
\begin{figure}[h!]
	\centering
	\includegraphics[width=15cm,height=7cm]{imagenes/desarrollo/diagramas/BPMN_REGISTRAR_CUENTA.png}
	\caption{Proceso de registro de nuevo usuario.}
	\label{fig:regsuccess}
\end{figure}


Funciona de manera similar a la Lambda del Login. Recibe tres parámetros en formato JSON a través de un request con método POST, estos parámetros son: nombre de usuario, contraseña y primer nombre. \par
Esta función se encarga de tomar estos datos y registrar un nuevo usuario con ellos. La contraseña que se guarda en la base de datos es el hash criptográfico con SHA-512, la cual será usada en el primero proceso de validación del inicio de sesión. En la figura 4.33 se observa el flujo de información para esta función.
\begin{figure}[h!]
	\centering
	\includegraphics[width=15cm,height=3.5cm]{imagenes/desarrollo/arquitectura/REGISTER_SUCCESS.png}
	\caption{Flujo de información de registro de usuario.}
	\label{fig:regsuccess}
\end{figure}
\par

\subsubsection{Guardar escenario}
El objetivo de ésta función es poder almacenar los escenarios que vayan generando los usuarios para que después puedan ser visualizados desde la aplicación. En la figura 4.34 se puede apreciar el modelado del proceso de almacenamiento de escenario. \par
\begin{figure}[h!]
	\centering
	\includegraphics[width=13cm,height=7cm]{imagenes/desarrollo/diagramas/BPMN_STORESC.png}
	\caption{Proceso de guardado de escenario.}
	\label{fig:regsuccess}
\end{figure}

La función recibe como parámetros el id del usuario al que se va a asociar el escenario, el nombre del escenario y un array de imágenes en formato \textbf{Base64}. una vez recibidos estos atributos, la función se encarga de convertir las imágenes en formato binario para después guardarlas en un Bucket de AWS S3 (Simple Storage Service) que no es más que un contenedor de archivos. Una vez guardadas las imágenes en el Bucket, se obtienen los nombres y las URL de las imágenes guardadas. Los nombres sirven para guardar el escenario en una base de datos, y posteriormente poder recuperarlos del Bucket a través del nombre. Las URL de las imágenes se envían con el objetivo de poder visualizar el escenario guardado en el dispositivo móvil obteniendo las imágenes directamente del almacenamiento en la nube. El proceso de devolver las URL de las imágenes será implementado también en la función de "VER ESCENARIO" con el objetivo de hacer que la respuesta a la petición sea más ligera, y con ello más rápida. El flujo de ésta función se puede ver en la figura 4.35.

\begin{figure}[H]
	\centering
	\includegraphics[width=15cm,height=3.5cm]{imagenes/desarrollo/arquitectura/STORESC_SUCCESS.png}
	\caption{Flujo de información de guardado de escenario.}
	\label{fig:regsuccess}
\end{figure}
		\section{Iteración IV}
\subsection{Resumen}
En esta iteración se modelaron dos muebles más, tres botones para cambiar de modelo e implementamos un botón para tomar fotografía del escenario.
\subsection{Desarrollo}
Se modelaron dos muebles más, utilizando la misma herramienta, obteniendo los mismos archivos y realizando el proceso de las iteraciones I y II.

En la parte inferior de la aplicación se encuentran tres diseños de muebles, que funcionan como botones para seleccionar el objeto 3D que aparecerá en pantalla.

En la parte superior derecha de la aplicación tenemos ubicado un botón el cual toma una screenshot de la escena y la guarda en el dispositivo.

\begin{figure}[H]
	\centering
	\includegraphics[width=4cm,height=8cm]{imagenes/iteraciones/all.png}
	\caption{Aplicación: modelos en la parte inferior y botón de tomar foto parte superior}
	\label{fig:aplicacion}
\end{figure} 

Al presionar el botón ``tomar fotografía" se guardará una imagen de la escena, en formato JPG y aparecerá un mensaje para abrir la ubicación de la imagen.  

\begin{figure}[H]
	\begin{minipage}{0.48\textwidth}
		\centering
		\includegraphics[width=5cm,height=10cm]{imagenes/iteraciones/save1.png}
		\caption{Pantalla de visualización de fotografías}
		\label{fig:save}
	\end{minipage}\hfill
	\begin{minipage}{0.48\textwidth}
		\centering
		\includegraphics[width=5cm,height=10cm]{imagenes/iteraciones/foto.jpg}
		\caption{Pantalla de registro nuevo usuario}
		\label{fig:foto}
	\end{minipage}\hfill
\end{figure}
Las interfaz que se muestra es temporal, se realizó de esta manera para probar la funcionalidad del aplicación al cumplir con algunos requerimientos funcionales, la interfaz final será como se muestra en los Mockups en la sección de \textbf{Anexos}.

\subsection{Conclusión}
El resultado de esta iteración es el entregable que mostramos mostramos en la presentación de TT 1. \par
En este punto nos dimos cuenta que la decisión que tomamos de usar ARCore para el desarrollo del proyecto fue correcta. La plataforma funciona de acuerdo a las pruebas de contexto que realizamos al principio y nos ha servido perfectamente para el cumplimiento de algunos requerimientos funcionales planteados en la sección de \textbf{Análisis}. El desarrollo que involucra únicamente la implementación y uso de la realidad aumentada no fue tan problemático. \par Al momento de desarrollar las iteraciones, los desarrolladores de Sceneform SDK y ARCore SDK lanzaron nuevas versiones de cada software con correcciones, mejoras, mismas que serán actualizadas en el proyecto. 
\clearpage
		%%%%%%%   BIBLIOGRAFÍA   %%%%%%%
		\chapter{Anexos}

\section{Dieño de interfaces de usuario}
\begin{figure}[h!]
	\begin{minipage}{0.48\textwidth}
		\centering
		\includegraphics[width=4cm,height=8cm]{imagenes/Anexos/Mockup/1-Login.png}
		\caption{UI1 - Pantalla inicial de autenticación}
		\label{fig:analogo}
	\end{minipage}\hfill
	\begin{minipage}{0.48\textwidth}
		\centering
		\includegraphics[width=4cm,height=8cm]{imagenes/Anexos/Mockup/2-principalSinLogin.png}
		\caption{UI2 - Pantalla principal sin autenticación}
		\label{fig:analogo}
	\end{minipage}\hfill
\end{figure}

\begin{figure}[h!]
	\begin{minipage}{0.48\textwidth}
		\centering
		\includegraphics[width=4cm,height=8cm]{imagenes/Anexos/Mockup/3-MenuMuebles.png}
		\caption{UI3 - Menú desplegable de muebles (catalogo)}
		\label{fig:analogo}
	\end{minipage}\hfill
	\begin{minipage}{0.48\textwidth}
		\centering
		\includegraphics[width=4cm,height=8cm]{imagenes/Anexos/Mockup/4-principalLogin.png}
		\caption{UI4 - Pantalla principal tras autenticación}
		\label{fig:analogo}
	\end{minipage}\hfill
\end{figure}

\begin{figure}[h!]
	\begin{minipage}{0.48\textwidth}
		\centering
		\includegraphics[width=4cm,height=8cm]{imagenes/Anexos/Mockup/5-GuardarEscenario.png}
		\caption{UI5 - Pantalla de almacenamiento de escenario}
		\label{fig:analogo}
	\end{minipage}\hfill
	\begin{minipage}{0.48\textwidth}
		\centering
		\includegraphics[width=4cm,height=8cm]{imagenes/Anexos/Mockup/6-MisEscenarios.png}
		\caption{UI6 - Pantalla de visualización de escenarios }
		\label{fig:analogo}
	\end{minipage}\hfill
\end{figure}

\begin{figure}[h!]
	\begin{minipage}{0.48\textwidth}
		\centering
		\includegraphics[width=4cm,height=8cm]{imagenes/Anexos/Mockup/7-Foto.png}
		\caption{UI7 - Pantalla de visualización de fotografías}
		\label{fig:analogo}
	\end{minipage}\hfill
	\begin{minipage}{0.48\textwidth}
		\centering
		\includegraphics[width=4cm,height=8cm]{imagenes/Anexos/Mockup/8-Registro.png}
		\caption{UI8 - Pantalla de registro nuevo usuario}
		\label{fig:analogo}
	\end{minipage}\hfill
\end{figure}

\begin{figure}[h!]
	\begin{minipage}{0.48\textwidth}
		\centering
		\includegraphics[width=4cm,height=8cm]{imagenes/Anexos/Mockup/9-RecuperarP.png}
		\caption{UI9 - Pantalla de recuperación de contraseña 1}
		\label{fig:analogo}
	\end{minipage}\hfill
	\begin{minipage}{0.48\textwidth}
		\centering
		\includegraphics[width=4cm,height=8cm]{imagenes/Anexos/Mockup/10-RecuperarP2.png}
		\caption{UI10 - Pantalla de recuperación de contraseña 2}
		\label{fig:analogo}
	\end{minipage}\hfill
\end{figure}

\begin{figure}[h!]
	\begin{minipage}{0.48\textwidth}
		\centering
		\includegraphics[width=4cm,height=8cm]{imagenes/Anexos/Mockup/11-ActualizarP.png}
		\caption{UI11 - Pantalla de actualización de contraseña}
		\label{fig:analogo}
	\end{minipage}\hfill
	\begin{minipage}{0.48\textwidth}
		\centering
		\includegraphics[width=4cm,height=8cm]{imagenes/Anexos/Mockup/12-VerEscenario.png}
		\caption{UI12 - Pantalla de visualización de imágenes de escenario }
		\label{fig:analogo}
	\end{minipage}\hfill
\end{figure}   
	%%%%%%%   BIBLIOGRAFÍA   %%%%%%%
	\begin{thebibliography}{X}
	
	\bibitem{B15} \textsc{Hsu, Pei-Hsien}, \textsc{Huang, Sheng-Yang} y \textsc{Lin, Bao-Shuh} "Smart-Device-Based Augmented Reality (SDAR) Models to Support Interior Design: Rethinking “Screen” in Augmented Reality", National Chiao Tung University, Taiwan
	
	\bibitem{B01} \textsc{Montes de Oca, Irina} y \textsc{Risco, Lucía},
	"Apuntes de diseño de interiores",
	\textit{Principios básicos de escalas. espacios, colores y más}, primera edicion,
	ECOE EDICIONES
	
	\bibitem{B02} "FAQ for designers", \textit{Whats is Interior Design}, Interior Design Legislative Coalition of Pennsylvania (IDLCPA), 2011. Recuperado de: https://www.idlcpa.org/forms/resources/FAQforDesigners.pdf
	
	\bibitem{B26} Kathryn, T.,(2000), feng shui habitación por habitación (12ª ed.). ES, Urano.	
	
	\bibitem{B24} Harris, C., (2006), dictionary of architecture and construction (4ª ed.). New York, EU, McGraw-Hill.
	
		
	\bibitem{B13} \textsc{Viola, Romina} (11 de mayo de 2016), "Cómo Elegir la Paleta de Colores", \textit{Parte I: Entender el Color}. Spanish Community Champion | Piktochart. [Online] Recuperado de: https://piktochart.com/es/blog/como-elegir-la-paleta-de-colores-parte-entender-el-color/
	
	
	
	\bibitem{B04} \textsc{Siltanen, Sanni} "Diminished reality for augmented reality interior design", Springer-Verlag Berlin Heidelberg 2015. Publicado el: 30 de Noviembre de 2015. DOI: 10.1007/s00371-015-1174-z
	DOI 10.1007/978-3-319-54502-8
	
	\bibitem{B05} \textsc{Donggang Yu1, Jesse Sheng Jina},
	"A Useful Visualization Technique: A Literature Review for Augmented Reality and its Application, limitation and future direction", primera edicion,
	School of Design, Communication and Information Technology, The University of Newcastle, Callaghan, NSW 2308, Australia
	
	
	
	\bibitem{B27} \textsc{Furht, Borko} "Handbook of Augmented Reality",Department of Computer and Electrical Engineering and Computer Science,Florida Atlantic University (),Glades Road 777 33431 Boca Raton, Florida USA,DOI 10.1007/978-1-4614-0064-6
	
	\bibitem{B22} \textsc{Peddie, Jon} "Augmented Reality Where we will all live", Springer International Publishing AG 2017. Publicado el: 19 de Abril del 2017. DOI 10.1007/978-3-319-54502-8 
	
	\bibitem{B09} \textsc{Monteiro, Paula} and \textsc{Nagele, Aleksandra}, "Wikitude" \textsc{Company Overview}, Recuperado de: https://s3-eu-west-1.amazonaws.com/wikitude-web-hosting/static-website/2017/07/01728364/Media+page+presentation+-+Wikitude.pdf
	
	
	
	\bibitem{B16} "Wikitude Products", 29 de agosto de 2018. Recuperado de: https://www.wikitude.com/products/wikitude-sdk/
	
	
	\bibitem{B20} "ARKit Developer", 2 de Septiembre de 2018. Recuperado
	https://developer.apple.com/arkit/
	
	\bibitem{B21} "ARKit Documentation", 3 de Septiembre de 2018. Recuperado
	https://developer.apple.com/documentation/arkit/
	
	
	\bibitem{B12} \textsc{Grahn, Ivar}, "The Vuforia SDK and Unity3D Game Engine", \textit{Evaluating performanceo on Android Devices}. Linköping University | Department of Computer and Information, Science Bachelor thesis, 16 ECTS, Computer Science 2017, LIU-IDA/LITH-EX-G–17/059–SE
	
	
	\bibitem{B14} "ARCore Overview", 2 de agosto de 2018. Recuperado de: https://developers.google.com/ar/discover/
	\\
	
	
	%
	
	\bibitem{B03} \textsc{Takahashi, Yoshiyuki} y \textsc{Mizumura, Hiroko},
	"Augmented Reality Based Environment Design Support System for Home Renovation", Toyo  University, Department of Human Environment Design, Faculty of Human Life Design, Oka 48-1, Asaka-shi, Saitama, 351-8510 Japan 
	
	
	
	\bibitem{B06} \textsc{Buerli, Mike} and \textsc{Misslinger, Stefan}, "Introducing ARKit", \textit{Augmented Reality for iOS. Session 602} Recuperado de: https://devstreaming-cdn.apple.com/videos/wwdc/2017/602pxa6f2vw71ze/602/602
	\_introducing\_arkit\_augmented\_reality\_for\_ios.pdf?dl=1
	
	\bibitem{B07} \textsc{Strandmark, Petter}, "Augmented reality with the ARToolKit", Recuperado de: http://www.maths.lth.se/matematiklth/personal/petter/rapporter/artoolkit4.pdf
	
	\bibitem{B08} \textsc{Fuster Andújar, Francisco de Asís}, "Aplicación Android de realidad aumentada para mostrar imágenes históricas de lugares turísticos de interés", Tesis. Escola Tècnica Superior d’Enginyeria Informàtica Universitat Politècnica de València, Valencia, España, 2014.
	
	
	\bibitem{B10} \textsc{Asrul Sani, Nisfu}, "Google ARCore", Lab Langit 9 - PENS 20-22 de noviembre de 2017. Recuperado de: http://dhoto.lecturer.pens.ac.id/training/arcore/SONI-ARCORE.pdf
	
	\bibitem{B11} \textsc{Kaleda, Yuliya}, "Add Reality to App with ARCore". Recuperado de: https://downloads.ctfassets.net/2grufn031spf/7AyZDCxxIs0AoOgwOsy0m4/26f89f9d2
	f346dd3c6a55f5d3e2de96e/Yuliya\_Kaleda\_Add\_Reality\_To\_App\_with\_ARCore.pdf
	

	
	
	
	\bibitem{B18} "OpenGL Documentation", 30 de agosto de 2018. Recuperado de:
	https://www.opengl.org/documentation/
	
	\bibitem{B19} "OpenGL Glut", 30 de agosto de 2018. Recuperado
	https://www.opengl.org/resources/libraries/glut/spec3/spec3.html
	
    
	
	\bibitem{B23} "Vuforia Supported-devices", 4 de Septiembre de 2018. Recuperado
	https://library.vuforia.com/articles/Solution/vuforia-fusion-supported-devices.html
	
	\bibitem{B25} Medina, V. E., (2003). Forma y composición en la arquitectura descontructivista (Tesis doctoral). Escuela Técnica Superior de Arquitectura de Madrid, Madrid, Esp.
	
	
	\bibitem{B28} "The new power tool for home improvement", 22 de Octubre del 2018. Recuperado https://canvas.io/
	\bibitem{B29} "Amazon.com.mx" 22 de Octubre. Recuperado  de  2018 https://www.amazon.com.mx/ 
	\bibitem{B30} "Fingo. Furniture. Try before you buy!" 22 de Octubre 2018 https://itunes.apple.com/us/app/fingo.vybor-mebeli-katalog/id845105741
	\bibitem{B29} "Amazon.com.mx" 22 de Octubre. Recuperado  de  2018 https://www.amazon.com.mx/ 
	
	
\end{thebibliography}

	
	
\end{document}


%%%%%%%%%%%%%%%%%%%%%%%%%%%%%%%%%%%%%%%
%%%%%%%    FIN DEL DOCUMENTO    %%%%%%%
%%%%%%%%%%%%%%%%%%%%%%%%%%%%%%%%%%%%%%%
