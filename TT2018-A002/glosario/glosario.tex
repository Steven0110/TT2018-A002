\chapter{Glosario}
En esta sección se definen algunos conceptos importantes para una mejor comprensión del presente documento enfocados contexto de diseño de interiores.\par
\vspace{10mm}

\noindent
\textbf{Agilizar}. Hacer que un proceso sea terminado en una menor cantidad de tiempo.\par
\vspace{2mm}
\noindent
\textbf{Antropometría}. Es una rama de la antropología biológica encargada del estudio de las dimensiones del cuerpo humano y su relación con su entorno.\par
\vspace{2mm}
\noindent
\textbf{Complejidad}. Es la cualidad de aquello que es compuesto por un gran número de elementos relacionados entre sí.\par
\vspace{2mm}
\noindent
\textbf{Cotización}. Es un desglose que define el costo total de un producto o servicio así como el de los elementos que lo componen.\par
\vspace{2mm}
\noindent
\textbf{Efectivo}. Aquello que cumple con los resultados esperados.\par
\vspace{2mm}
\noindent
\textbf{Entorno}. Es el conjunto de elementos y circunstancias que rodean una cosa.\par
\vspace{2mm}
\noindent
\textbf{Escalabilidad}. Es la propiedad del software que le permite crecer sin afectar su funcionamiento y calidad.\par
\vspace{2mm}
\noindent
\textbf{Escenario}. Es el conjunto de imágenes, videos e información que describen a un inmueble en particular, con las características agregadas de la realidad aumentada.\par
\vspace{2mm}
\noindent
\textbf{Espacio}. Es el lugar físico donde se encuentran posicionados los elementos de algún inmueble.\par
\vspace{2mm}
\noindent
\textbf{Factura}. Es el documento oficial que detalla la compra de bienes y/o servicios. En México toda factura emitida debe tener un impuesto al valor agregado (IVA) correspondiente al 16\% del valor total de la factura.\par
\vspace{2mm}
\noindent
\textbf{Iteración}. Es una de las repeticiones que conforman a un proceso cíclico.\par
\vspace{2mm}
\noindent
\textbf{Lentitud}. Es la cualidad de los procesos realizados en una gran cantidad de tiempo.\par
\vspace{2mm}
\noindent
\textbf{Modelo renderizado}. Es un objeto virtual en 3D procesado y visible a través de algún medio de visualización como smartphones y computadoras.\par
\vspace{2mm}
\noindent
\textbf{Optimización}. Es la mejora de un proceso para lograr su completitud en un menor tiempo y con un menor número de recursos.\par
\vspace{2mm}
\noindent
\textbf{Plano}. Es una superficie bidimensional sobre la cual son posicionados elementos virtuales en la realidad aumentada.\par
\vspace{2mm}
\noindent
\textbf{Plataforma de software}. Es el conjunto de herramientas relacionadas entre sí que sirven como base para el desarrollo de otros sistemas de software.\par
\vspace{2mm}
\noindent
\textbf{Proyecto}. Es el conjunto de escenarios relacionados a un usuario en particular, que también tiene información del mismo.\par
\vspace{2mm}
\noindent
\textbf{Prueba de contexto}. Es la prueba de alguna herramienta realizada para comprobar el desempeño en alguna de sus funciones.\par
\vspace{2mm}
\noindent
\textbf{Tap}. Es la acción realizada al poner y retirar el dedo rápidamente sobre una pantalla táctil.\par