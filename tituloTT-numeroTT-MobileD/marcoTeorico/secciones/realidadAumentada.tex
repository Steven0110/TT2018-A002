\section{Realidad Aumentada AR}

La mayoría de las veces asociamos los términos de realidad aumentada con la realidad virtual como si fueran lo mismo, sin embargo existen grandes motivos para detallar sus diferencias de software, hardware y de la interacción con los usuarios. \\.\par 

La realidad aumentada (AR) nos permite sobreponer objetos en tercera dimensión sobre una imagen en tiempo real, obteniendo una mezcla de lo real con lo virtual, mejorando la percepción del mundo real de un usuario. Estas imagenes en tiempo real las podemos obtener con las camaras de smartphones, tablets, smart glasses y computadoras. [1] \\. \par

Por otro lado la realidad virtual (VR) se define como un sistema informático que genera en tiempo real representaciones de una realidad es decir, un mundo virtual donde puede llegar a existir una interacción con el usuario. Muchas de estas simulaciones requieren de gafas de VR compatibles con smartphones o gafas especiales como el Gear VR desarrollado por Samsumg en colaboración con Oculus VR. [2][3] \\. \par

A diferencia de la VR, la AR es una tecnología que complementa  la percepción  e interacción  con el mundo real y permite al usuario estar en un entorno aumentado con información generada por computadora. [1] \\. \par

Actualmente, existen dos grandes representantes en estas tecnologías. Tal vez por criterios de marketing diríamos que Google Glass es el representante de la realidad aumentada, mientras que el Oculus Rift de Facebook sería el representante de la realidad virtual. [4]
