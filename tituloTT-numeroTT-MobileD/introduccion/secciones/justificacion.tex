\section{Justificación}
Cuando nos cambiamos de hogar, inevitablemente tenemos que afrontarnos con la tarea de decorar las habitaciones que hay en él. En éste punto, hacerlo no resulta tan complicado dado que partimos de una habitación vacía y ésta se convierte en un lienzo en blanco para nuestra imaginación. Al no haber objetos presentes, la percepción espacial de quien decora no se ve afectada, de tal forma que éste escenario facilita el diseño de interiores. Desafortunadamente no siempre tenemos la oportunidad de decorar una habitación cuando ésta se encuentra vacía, pues normalmente ya hay muebles y objetos decorativos en ella, entonces el proceso se resume a agregar nuevos objetos. Si nosotros escogemos un mueble que se ve agradable a simple vista, puede que, al momento de colocarlo en la habitación, no se vea tan bien y no se encuentre en armonía con los demás objetos decorativos. Incluso puede que el objeto ni siquiera quepa en el lugar donde se había planeado su posición y sea necesario reordenar la habitación, lo cual es cansado, dependiendo del peso y la posición de los muebles.\par 
Aunado a esto, puede llegar el punto donde quien decora la habitación, al final ya no desee el mueble, y realice un proceso de devolución de producto, si es que la tienda donde lo compró lo permite. Entonces la tienda pasa al domicilio donde se encuentre el producto para recogerlo o el usuario va a la tienda a entregarlo. De cualquier forma, se traduce en una pérdida económica y de tiempo.\par 
Todas éstas consecuencias se podrían evitar si realizamos un diseño de interiores efectivo, es decir, podamos saber si un mueble se va a ver bien en nuestra sala o comedor incluso antes de comprarlo. También facilitaría el proceso, saber las propiedades del producto, como pueden ser el peso y sus dimensiones exactas para determinar si se requiere un flete o no.\par 
Por lo anterior, el diseño de interiores  para las personas se vuelve complejo, tardado y tedioso, lo cual puede provocar pérdidas económicas y de tiempo por parte del cliente que compra un mueble y/o por parte de la tienda si se efectúa un proceso de devolución de producto dañando el prestigio de la tienda o sucursal asociada a la venta de estos muebles u objetos.
